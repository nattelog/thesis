\chapter{Discussion}
\label{cha:discussion}

This chapter contains the following sub-headings.

\section{Results}
\label{sec:discussion-results}

The results are satisfactory and provides a good image on how different
configurations of the gateway affect performance. There are however details in
the production method of the data that could be improved. Each data point is
the result of a single test scenario. Some diagrams have outliers caused by
small delays and hickups in the network and the operating system during the
test. If every data point instead was the average value of several test
scenarios with the same configuration, smoother diagrams would have been
obtained.

Regarding the choice of data. There were a total of 18 diagrams that could have
been presented in the result chapter. The most interesting ones were chosen and
they were selected on the criterias:

\begin{enumerate}
    \item They had unique outcomes. Only one of several diagrams that show the
        same result was chosen.
    \item They were possible to explain within the time frame of the project.
    \item They show significant difference between at least one of the event propagation
        models.
\end{enumerate}

The choice of values for the parameters that varied in the tests required some
amount of testing in order to get sensable results. Too small steps lead either
to no significant change in performance or too many steps in order to see some
change in performance. 10 steps took 2-3 hours to fully test all possible
event propagation models. Many tests were redone to calibrate best possible
step size and value window of the parameter. For instance, changing CPU
intensity from 0.01 to 0.1 with 0.01 steps produced very small change in
performance, compared to changing the window to 0.1-0.5.

\section{Method}
\label{sec:discussion-method}

Some new ideas are proposed in this study: the three event propagation models
and the abstract gateway. They have all been verified to work as expected
\textit{in this context}. There are limitations that have not been thouroughly
tested yet, e.g. the push-based dispatching approach where devices actively
push data to the gateway. There is currently no parameter to the abstract
gateway that takes active/passive devices into account. This should be further
developed.

\section{The work in a wider context}
\label{sec:work-wider-context}

There must be a section discussing ethical and societal
aspects related to the work. This is important for the authors
to demonstrate a professional maturity and also for achieving
the education goals. If the work, for some reason, completely
lacks a connection to ethical or societal aspects this must be
explicitly stated and justified in the section Delimitations in
the introduction chapter.

In the discussion chapter, one must explicitly refer to sources
relevant to the discussion.
