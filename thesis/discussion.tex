\chapter{Discussion}
\label{cha:discussion}

This chapter contains the following sub-headings.

\section{Results}
\label{sec:discussion-results}

The results are satisfactory and provides a good image on how different
configurations of the gateway affect performance. There are however details in
the production method of the data that could be improved. Each data point is
the result of a single test scenario. Some diagrams have outliers caused by
small delays and hiccups in the network and the operating system during the
test. If every data point instead was the average value of several test
scenarios with the same configuration, smoother diagrams would have been
obtained.

Regarding the choice of data. There were a total of 18 diagrams that could have
been presented in the result chapter. The most interesting ones were chosen and
they were selected on the criterions:

\begin{enumerate}
    \item They had unique outcomes. Only one of several diagrams that show the
        same result was chosen.
    \item They were possible to explain within the time frame of the project.
    \item They show significant difference between at least one of the event
        propagation models.
\end{enumerate}

The choice of values for the parameters that varied in the tests required some
amount of testing in order to get sensible results. Too small steps lead either
to no significant change in performance or too many steps in order to see some
change in performance. 10 steps took 2-3 hours to fully test all possible
event propagation models. Many tests were redone to calibrate best possible
step size and value window of the parameter. For instance, changing CPU
intensity from 0.01 to 0.1 with 0.01 steps produced very small change in
performance, compared to changing the window to 0.1-0.5.

\section{Method}
\label{sec:discussion-method}

Some new ideas are proposed in this study: the three event propagation models
and the abstract gateway. They have all been verified to work as expected
\textit{in this context}. There are limitations that have not been thoroughly
tested yet, e.g. the push-based dispatching approach where devices actively
push data to the gateway. There is currently no parameter to the abstract
gateway that takes active/passive devices into account. This should be further
developed.

Regarding the work $\Lambda_j$ discussed in Section
\ref{sec:the_event_handler}. The simplification that all events induce the same
amount of work is not necessarily correct in practice. Wu et al.
\cite{wu2011m2m} discuss how large numbers of devices will be available on the
market and if different types of devices are connected to the same gateway, it
is obvious that they will induce different amount of work on it. Also, as noted
in the same section, there can be dependency between CPU work and I/O work. The
implementation done in this study has entirely designed the CPU- and I/O work
to be independent on each other.

\section{The work in a wider context}
\label{sec:work-wider-context}

Internet of Things is not only applicable for industries with economic gain in
mind, but also for institutions like healthcare, education and public safety
\cite{gubbi2013internet}. Improving the important gateway component of the IoT
infrastructure and understanding its performance characteristics can help the
development of IoT applications in those important areas of society.
