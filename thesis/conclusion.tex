\chapter{Conclusion}
\label{cha:conclusion}

To understand what internal and external properties affects the functionality
and performance of the IoT gateway, a combination of observations of
state-of-the-art technologies and consulting from experienced professionals led
to a list of common parameters in most IoT gateway systems. These parameters
were used to develop a theoretical model called the abstract gateway, which is
proposed to model \textit{any} type of gateway application. This model laid the
foundation to a configurable gateway implementation and a performance test
system. The test system was used to test the performance of different
configurations of the gateway implementation. The internal and external
properties of the gateway are:

\begin{enumerate}

    \item The number of devices.
    \item The frequency of generated events on each device.
    \item The delay added to each request on the network.
    \item The event propagation model of the dispatcher.
    \item The event propagation model of the event handler.
    \item The CPU intensity induced by each event.
    \item The I/O intensity induced by each event.
    \item The number of CPU cores on the gateway machine.

\end{enumerate}

The abstract gateway has two major building blocks: the dispatcher and the
event handler. The dispatcher handles incoming events from the devices
communicating with the gateway and dispatch them forward to the event handler.
The event handler performs some CPU- and I/O intensive work on each event.
Three event propagation models are proposed that describes the fundamental
architectures of the dispatcher and the event handler. The serial event
propagation model communicates with devices and handles events serially, i.e.
one device at a time. The preemptive model can communicate with several devices
and handle several events concurrently on different threads and the operating
system scheduler interleaves between the working threads implicitly. The
definition of the cooperative model states that CPU and I/O work induced by
each event can \textit{explicitly} interleave and make room for other work. The
cooperative model is implemented with libuv and allows multiple devices and
events to be handled in parallel while still only running on one major thread.

Increasing the number of devices will increase the response time drastically
for the serial and cooperative event handlers compared to the preemptive ones.
The cooperative dispatcher gives much better response time when network delay
is increased, however. For single core CPUs, all event propagation models
perform equally in terms of throughput when CPU intensity is increased. The
cooperative dispatcher combined with a cooperative event handler creates worst
response time, despite the number of cores. For I/O intensive work, the
cooperative approaches perform best in terms of throughput.

The best approach to implement libuv in this context is to model the gateway as
a state machine. As libuv uses callback functions to let the user know when
asynchronous I/O work is done, the callbacks can be seen as vertices in the
state machine. Whenever an asynchronous I/O request is initiated in libuv, the
user can explicitly choose other tasks to do while the I/O request is working.
This makes libuv a good choice to implement the cooperative dispatcher. The
results show that the cooperative event handler is not ideal for CPU intensive
events. libuv is primarily single-threaded; it is not the best alternative for
implementing cooperative event handlers if the CPU intensity of events is high.
However, utilizing the built-in threadpool in libuv one can built a preemptive
event handler on top of a cooperative dispatcher. The results of the
performance tests shows that it performs better or equal to the next best event
propagation model in all test cases.

\section{Future work}

The preemptive dispatcher was not developed and it would be interesting to do
so in future work. The performance of a preemptive dispatcher would probably
match the cooperative one, at least on multicore systems. This study only
focused on passive devices that needed to be polled in order to retrieve their
events. The alternative are active devices that push data to the gateway by
themselves. This approach would be interesting to study as well as it resembles
typical client-server architectures that are common to web solutions.
