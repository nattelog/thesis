\chapter{Method}
\label{cha:method}

The aim of this study is to compare libuv to current state of the art
techniques for applications in IoT. Previous attempts have been made to prove
how certain programming languages perform better when used in a reactive
context. Terber \cite{terber2017function} discusses the lack of
function-oriented software decomposition for reactive software. With an
industrial application as context, he replaces legacy code with code written in
the \textit{Cèu} programming language, reaching the conclusion that Cèu
preserves fundamental software engineering principles and is at the same time
able to fullfill resource limitations in the system. Jagadeesan et al
\cite{jagadeesan1996formal} performs a similar study but with a different
language: \textit{Esterel}. They reimplement a component in a telephone
switching system and reach the conclusion that Esterel is better suited for
analysis and verification for reactive systems.

\section{Development methodology}

This entire study will be conducted with Scrum as its backbone. Both the
writing of the thesis and the development of the applications to test will
happen in sprints. The project's backlog will initially be the fundamental
requirements of a master's thesis (based on requirements from Linköping
University) and the research questions. The forms of its user stories will
resemble traditional issues or requirements, but their scope can be wide and
their acceptance criteria abstract. Every week they undergo refinement and
abstract stories will be split into concrete ones as new knowledge about them
is acquired during the project. For instance, a starting user story (or issue)
will be \textit{"write the Results chapter"}. Initially this story is very
large, it is hard to do it in one sprint and it is hard to know where to start.
As time progress and the application to test have been developed and tests have
been conducted, the story can be split into more precise issues like
\textit{"present the developed application"} and \textit{"present a diagram
with the test results"}. These issues are (subjectively) easier to do in one
sprint and it is also easier to know when they are finished.

The author will take on all three traditional Scrum roles: product owner, Scrum
master and team member. Other stakeholders of the project are representatives
from Attentec, the examiner and a mentor from Linköping University. Each sprint
will have a length of 2 weeks and the sprint planning will, unlike the
traditional sprint planning \cite{sims2012scrum} where only the product owner
and team members are present, include a mentor from Attentec to help plan the
upcoming sprint. At the end of each sprint the current status of the project
will be presented to the stakeholders.

\section{Literature study}

A major literature study will be conducted prior and during the implementation.
The theoretical framework will be vindicated in this phase to support claims
and form a general direction of the entire study. Multiple databases will be
queried in order to find interesting material from previous research. Mainly
the online library hosted by \textit{Linköping
University}\footnote{\url{www.bibl.liu.se}} will be used since it allow access
to material otherwise unviewable due to institutional login requirements. Query
results from this library is a collection of query results from other research
databases such as \textit{ACM Digital Library}\footnote{\url{dl.acm.org}},
\textit{ProQuest Ebook Central}\footnote{\url{ebookcentral.proquest.com}} and
\textit{IEEE Xplore Digital Library}\footnote{\url{ieeexplore.ieee.org}}; so it
acts as a gateway to a global collection of scientific research.

The \textit{three-pass approach} presented by Keshav \cite{keshav2007read} will
be used as a basic approach to find interesting material. It helps the reader
grasp the paper's content in three \textit{passes}. The first pass' purpose is
to give the reader an overview of the paper. The title, abstract, introduction,
headings, sub-headings, conclusions and references are read. This information
should help the reader understand the paper's category and context and help
decide whether to continue read this paper or leave it. If the reader choose to
continue read it, the second pass starts. Here the paper is read more
thoroughly. The figures and diagrams are examined and after this pass the
reader should be able to summarize the paper, with leading evidence, to someone
else. The purpose of the third pass is to fully understand the paper. By making
the same assumptions as the author, the paper is virtually re-created. It helps
identify the true innovations of the paper, as well as the hidden failures.

\section{Implementation}

The main approach this study will undertake to answer its question is to:

\begin{enumerate}
  \item find a state of the art reactive system in the industry with an
    appropiate level of complexity
  \item create a specification of the system
  \item reimplement the system with libuv
  \item create and conduct tests
  \item apply and evaluate the maintainability metrics on both systems
  \item present the results
\end{enumerate}

With help from Attentec, a multi-sensor monitor application used by one of
their clients will be specified and its source code will be used to compare its
software maintainability to a reimplementation with libuv. If no appropiate
system can be found in the industry, Attentec will aid in creating a
specification for a similar system as well as a state of the art technique to
implement it. The specification will follow the same pattern presented by Ardis
et al \cite{ardis1996framework}.

Even though libuv is written in C, C++ will be used as the main programming
language for the libuv implementation. With the dynamic programming style C++
offers with classes and templates, it will be more suitable for an
implementation that hopefully will attract web developers.

A testing environment will be created to simulate an IoT context where multiple
sensors are connected to the monitor application. A software testing suite will
also be setup to create unit tests.

\section{Evaluation}

Same software maintainability metrics will be applied on the system found in
the industry and the reimplementation of it. The results of the metrics will be
compared and presented.
