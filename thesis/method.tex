\chapter{Method}
\label{cha:method}

The aim of this study is to present the differences in performance and
maintainability three event-driven architectures constitutes. Previous attempts
have been made to prove how certain programming languages perform better when
used in a reactive context. Terber \cite{terber2017function} discusses the lack
of function-oriented software decomposition for reactive software. With an
industrial application as context, he replaces legacy code with code written in
the \textit{Cèu} programming language, reaching the conclusion that Cèu
preserves fundamental software engineering principles and is at the same time
able to fullfill resource limitations in the system. Jagadeesan et al
\cite{jagadeesan1996formal} performs a similar study but with a different
language: \textit{Esterel}. They reimplement a component in a telephone
switching system and reach the conclusion that Esterel is better suited for
analysis and verification for reactive systems.

\section{The big picture}

An IoT architecture can be split up into a sensing domain, a network domain and
an application domain \cite{chen2011brief}. The IoT gateway resides in the
network domain and is decoupled from whatever application the data from the
sening domain is used for.

\section{Two customer cases}

Before presenting the architectural models used in this study, it is of
importance to understand how they came about. Among the customers of Attentec,
two were of interest for this study with each having implemented an IoT gateway
suitable for its operating context. \textit{Customer A} uses their IoT gateway
to monitor industrial batteries used in forklifts. The main idea behind the
event propagation model is that the gateway polls each battery for events and
process them according to its specification (which is unrelevant here).
Referring to data propagation in reactive languages (see section
\ref{sec:evaluation-model}), one can view the event propagation as
\textit{pull-based}, i.e. the gateway pulls events from the batteries whenever
it finds it necessary. The batteries are \textit{passive}, i.e. they do not
emit any events on their own, only when asked to by the gateway. Figure
\ref{fig:pull-based} illustrates this.

\begin{figure}[h!]
    \centering
    \begin{tikzpicture}[auto]
        \node [block] (device_1) {Device 1};
        \node [block, right of=device_1, node distance=3cm] (device_2) {Device 2};
        \node [block, right of=device_2, node distance=3cm] (device_n) {Device n};
        \node [block, below of=device_2, node distance=2.5cm] (gateway) {Gateway};

        \path [->] (device_1) edge [bend left=5] (gateway);
        \path [->] (gateway) edge [bend left=5] (device_1);
        \path [->] (device_2) edge [bend left=5] (gateway);
        \path [->] (gateway) edge [bend left=5] (device_2);
        \path [->] (device_n) edge [bend left=5] (gateway);
        \path [->] (gateway) edge [bend left=5] (device_n);
    \end{tikzpicture}
    \caption[Pull-based event propagation]{Pull-based event propagation with
    passive devices. The gateway pulls events from the devices by requesting
    them first.}
    \label{fig:pull-based}
\end{figure}

\textit{Customer B} on the other hand uses a \textit{push-based} approach to
handle event propagation in their gateway. The gateway hosts a REST API to
serve IoT devices. This means that the devices must be \textit{active}, i.e.
they emit events whenever they find it necessary. See Figure
\ref{fig:push-based} for an illustration. Note that not as many communication
requests are necessary in this approach compared to the pull-based one. The
gateway does not need to know about how many devices there are or what their
addresses are, but the devices need to know the address of the gateway. In
contrast, the pull-based approach requires the gateway to know about each
device, but the device does not need to know anything about the gateway.

\begin{figure}[h!]
    \centering
    \begin{tikzpicture}[auto]
        \node [block] (device_1) {Device 1};
        \node [block, right of=device_1, node distance=3cm] (device_2) {Device 2};
        \node [block, right of=device_2, node distance=3cm] (device_n) {Device n};
        \node [block, below of=device_2, node distance=2.5cm] (gateway) {Gateway};

        \path [line] (device_1) -- (gateway);
        \path [line] (device_2) -- (gateway);
        \path [line] (device_n) -- (gateway);
    \end{tikzpicture}
    \caption[Push-based event propagation]{Push-based event propagation with
    active devices. Events are pushed from the devices to the gateway that
    listens for incoming requests.}
    \label{fig:push-based}
\end{figure}

\section{Three event propagation models}

Taking inspiration from the three task management approaches presented by Adya
et al \cite{adya2002cooperative} and described in Section
\ref{sec:task-management}, three event propagation models are proposed:
\textit{serial}, \textit{preemptive} and \textit{cooperative event
propagation}. Before presenting a formal definition, an analogy might suite
well here. Imagine a cafe that serves coffee. There are several customers in
line and the barista takes the order from the next customer in line, brews the
coffee and serves it. The coffee brewer can only brew one cup of coffee at a
time. This is repeated for each customer in line with the constraint that no
new order is accepted before the current order is served, see Figure
\ref{fig:cafe-serial}. This can be seen as a \textit{serial order handling};
only one order can be handled at a time. Imagine now that the cafe hires more
baristas and more coffee machines. There is still one line, but orders can be
handled concurrently, so the next customer in line does not necessarily have to
wait for the previous customer to be served before he can place his order, see
Figure \ref{fig:cafe-preemptive}. This can be seen as \textit{preemptive order
handling}. Now lets keep the brewers but let go all baristas except for one.
All customers place their order to the same barista which in turn start one of
the brewers. While the brewer is working, the next customer in line places his
order and the barista can start a second brewer, see Figure
\ref{fig:cafe-cooperative}. This is called \textit{cooperative order handling}.

\begin{figure}[h!]
    \centering
    \begin{tikzpicture}[node distance=5cm, auto, >=stealth']
        \node[] (brewer) {Brewer};
        \node[right of=brewer] (barista) {Barista};
        \node[right of=barista] (customers) {Customer line};

        \node[below of=brewer, node distance=7.5cm] (brewer_ground) {};
        \node[below of=barista, node distance=7.5cm] (barista_ground) {};
        \node[below of=customers, node distance=7.5cm] (customers_ground) {};

        \draw (brewer) -- (brewer_ground);
        \draw (barista) -- (barista_ground);
        \draw (customers) -- (customers_ground);

        \draw[<-] ($(barista)!1cm!(barista_ground)$) --
            node[above, scale=0.9, midway]{Place $Order_0$}
            ($(customers)!1cm!(customers_ground)$);
        \draw[<-] ($(brewer)!1.1cm!(brewer_ground)$) --
            node[above, scale=0.9, midway]{Start brewer}
            ($(barista)!1.1cm!(barista_ground)$);
        \node[draw=none, fill=white, scale=0.9] at
            ($(brewer)!1.5cm!(brewer_ground)$) {Brewing...};
        \draw[->] ($(brewer)!1.9cm!(brewer_ground)$) --
            node[above, scale=0.9, midway]{Coffee ready}
            ($(barista)!1.9cm!(barista_ground)$);
        \draw[->] ($(barista)!2cm!(barista_ground)$) --
            node[above, scale=0.9, midway]{Serve $Order_0$}
            ($(customers)!2cm!(customers_ground)$);

        \node[draw=none, fill=white, scale=0.9] at
            ($(customers)!2.5cm!(customers_ground)$)
            {Next customer};

        \draw[<-] ($(barista)!3cm!(barista_ground)$) --
            node[above, scale=0.9, midway]{Place $Order_1$}
            ($(customers)!3cm!(customers_ground)$);
        \draw[<-] ($(brewer)!3.1cm!(brewer_ground)$) --
            node[above, scale=0.9, midway]{Start brewer}
            ($(barista)!3.1cm!(barista_ground)$);
        \node[draw=none, fill=white, scale=0.9] at
            ($(brewer)!3.5cm!(brewer_ground)$) {Brewing...};
        \draw[->] ($(brewer)!3.9cm!(brewer_ground)$) --
            node[above, scale=0.9, midway]{Coffee ready}
            ($(barista)!3.9cm!(barista_ground)$);
        \draw[->] ($(barista)!4cm!(barista_ground)$) --
            node[above, scale=0.9, midway]{Serve $Order_1$}
            ($(customers)!4cm!(customers_ground)$);

        \node[draw=none, fill=white, scale=0.9] at
            ($(customers)!4.5cm!(customers_ground)$)
            {Next customer};

        \node[draw=none, fill=white, scale=0.9] at ($(brewer)!5cm!(brewer_ground)$)
            {...};
        \node[draw=none, fill=white, scale=0.9] at ($(barista)!5cm!(barista_ground)$)
            {...};
        \node[draw=none, fill=white, scale=0.9] at ($(customers)!5cm!(customers_ground)$)
            {...};

        \draw[<-] ($(barista)!6cm!(barista_ground)$) --
            node[above, scale=0.9, midway]{Place $Order_j$}
            ($(customers)!6cm!(customers_ground)$);
        \draw[<-] ($(brewer)!6.1cm!(brewer_ground)$) --
            node[above, scale=0.9, midway]{Start brewer}
            ($(barista)!6.1cm!(barista_ground)$);
        \node[draw=none, fill=white, scale=0.9] at
            ($(brewer)!6.5cm!(brewer_ground)$) {Brewing...};
        \draw[->] ($(brewer)!6.9cm!(brewer_ground)$) --
            node[above, scale=0.9, midway]{Coffee ready}
            ($(barista)!6.9cm!(barista_ground)$);
        \draw[->] ($(barista)!7cm!(barista_ground)$) --
            node[above, scale=0.9, midway]{Serve $Order_j$}
            ($(customers)!7cm!(customers_ground)$);

        \node[draw=none, fill=white, scale=0.9] at ($(brewer)!7.5cm!(brewer_ground)$)
            {...};
        \node[draw=none, fill=white, scale=0.9] at ($(barista)!7.5cm!(barista_ground)$)
            {...};
        \node[draw=none, fill=white, scale=0.9] at ($(customers)!7.5cm!(customers_ground)$)
            {...};
    \end{tikzpicture}
    \caption{A sequence diagram of a serial cafe order line. Each customer is
    served one at a time and no customer is served before the previous customer
    receives its order.}
    \label{fig:cafe-serial}
\end{figure}

\begin{figure}[h!]
    \centering
    \begin{tikzpicture}[node distance=3cm, auto, >=stealth']
        \node[] (brewern) {$Brewer_{n-1}$};
        \node[right of=brewern] (baristan) {$Barista_{n-1}$};
        \node[right of=baristan, node distance=1.5cm] (dots) {...};
        \node[right of=dots, node distance=1.5cm] (brewer0) {$Brewer_0$};
        \node[right of=brewer0] (barista0) {$Barista_0$};
        \node[right of=barista0] (customers) {Customer line};

        \node[below of=brewern, node distance=7cm] (brewern_ground) {};
        \node[below of=baristan, node distance=7cm] (baristan_ground) {};
        \node[below of=dots, node distance=7cm] (dots_ground) {};
        \node[below of=brewer0, node distance=7cm] (brewer0_ground) {};
        \node[below of=barista0, node distance=7cm] (barista0_ground) {};
        \node[below of=customers, node distance=7cm] (customers_ground) {};

        \draw (brewern) -- (brewern_ground);
        \draw (baristan) -- (baristan_ground);
        \draw (brewer0) -- (brewer0_ground);
        \draw (barista0) -- (barista0_ground);
        \draw (customers) -- (customers_ground);

        \draw[<-] ($(barista0)!1cm!(barista0_ground)$) --
            node[above, scale=0.9, midway]{Place $Order_0$}
            ($(customers)!1cm!(customers_ground)$);
        \draw[<-] ($(brewer0)!1.1cm!(brewer0_ground)$) --
            node[above, scale=0.9, midway]{Start brewer}
            ($(barista0)!1.1cm!(barista0_ground)$);
        \node[draw=none, fill=white, scale=0.9] at
            ($(brewer0)!1.5cm!(brewer0_ground)$) {Brewing...};
        \draw[->] ($(brewer0)!2.9cm!(brewer0_ground)$) --
            node[above, scale=0.9, midway]{Coffee ready}
            ($(barista0)!2.9cm!(barista0_ground)$);
        \draw[->] ($(barista0)!3cm!(barista0_ground)$) --
            node[above, scale=0.9, midway]{Serve $Order_0$}
            ($(customers)!3cm!(customers_ground)$);

        \node[draw=none, fill=white, scale=0.9] at
            ($(customers)!1.5cm!(customers_ground)$)
            {Next customer};

        \draw[<-, dashed] ($(dots)!2cm!(dots_ground)$) --
            node[above, scale=0.9, pos=0.4]{Place $Order_1$}
            ($(customers)!2cm!(customers_ground)$);

        \node[draw=none, fill=white, scale=0.9] at
            ($(customers)!3.5cm!(customers_ground)$)
            {Next customer};

        \node[draw=none, fill=white, scale=0.9] at ($(brewern)!4cm!(brewern_ground)$)
            {...};
        \node[draw=none, fill=white, scale=0.9] at ($(baristan)!4cm!(baristan_ground)$)
            {...};
        \node[draw=none, fill=white, scale=0.9] at ($(brewer0)!4cm!(brewer0_ground)$)
            {...};
        \node[draw=none, fill=white, scale=0.9] at ($(barista0)!4cm!(barista0_ground)$)
            {...};
        \node[draw=none, fill=white, scale=0.9] at ($(customers)!4cm!(customers_ground)$)
            {...};

        \draw[<-] ($(baristan)!4.5cm!(baristan_ground)$) --
            node[above, scale=0.9, midway]{Place $Order_j$}
            ($(customers)!4.5cm!(customers_ground)$);
        \draw[<-] ($(brewern)!4.6cm!(brewern_ground)$) --
            node[above, scale=0.9, midway]{Start brewer}
            ($(baristan)!4.6cm!(baristan_ground)$);
        \node[draw=none, fill=white, scale=0.9] at
            ($(brewern)!5.5cm!(brewern_ground)$) {Brewing...};
        \draw[->] ($(brewern)!6.4cm!(brewern_ground)$) --
            node[above, scale=0.9, midway]{Coffee ready}
            ($(baristan)!6.4cm!(baristan_ground)$);
        \draw[->] ($(baristan)!6.5cm!(baristan_ground)$) --
            node[above, scale=0.9, midway]{Serve $Order_j$}
            ($(customers)!6.5cm!(customers_ground)$);

        \draw[->, dashed] ($(dots)!5.5cm!(dots_ground)$) --
            node[above, scale=0.9, pos=0.41]{Serve $Order_{j-1}$}
            ($(customers)!5.5cm!(customers_ground)$);

        \node[draw=none, fill=white, scale=0.9] at ($(brewern)!7cm!(brewern_ground)$)
            {...};
        \node[draw=none, fill=white, scale=0.9] at ($(baristan)!7cm!(baristan_ground)$)
            {...};
        \node[draw=none, fill=white, scale=0.9] at ($(brewer0)!7cm!(brewer0_ground)$)
            {...};
        \node[draw=none, fill=white, scale=0.9] at ($(barista0)!7cm!(barista0_ground)$)
            {...};
        \node[draw=none, fill=white, scale=0.9] at ($(customers)!7cm!(customers_ground)$)
            {...};
    \end{tikzpicture}
    \caption{A sequence diagram of a preemptive cafe order line. There are
    $n$ baristas and brewers that can handle orders concurrently and
    customers does not necessarily have to wait for previous orders to be
    returned before they can place their own.}
    \label{fig:cafe-preemptive}
\end{figure}

\begin{figure}[h!]
    \centering
    \begin{tikzpicture}[node distance=3cm, auto, >=stealth']
        \node[] (brewern) {$Brewer_{n-1}$};
        \node[right of=brewern, node distance=1.5cm] (dots) {...};
        \node[right of=dots, node distance=1.5cm] (brewer0) {$Brewer_0$};
        \node[right of=brewer0] (barista) {Barista};
        \node[right of=barista] (customers) {Customer line};

        \node[below of=brewern, node distance=7cm] (brewern_ground) {};
        \node[below of=dots, node distance=7cm] (dots_ground) {};
        \node[below of=brewer0, node distance=7cm] (brewer0_ground) {};
        \node[below of=barista, node distance=7cm] (barista_ground) {};
        \node[below of=customers, node distance=7cm] (customers_ground) {};

        \draw (brewern) -- (brewern_ground);
        \draw (brewer0) -- (brewer0_ground);
        \draw (barista) -- (barista_ground);
        \draw (customers) -- (customers_ground);

        \draw[<-] ($(barista)!1cm!(barista_ground)$) --
            node[above, scale=0.9, midway]{Place $Order_0$}
            ($(customers)!1cm!(customers_ground)$);
        \draw[<-] ($(brewer0)!1.1cm!(brewer0_ground)$) --
            node[above, scale=0.9, midway]{Start brewer}
            ($(barista)!1.1cm!(barista_ground)$);
        \node[draw=none, fill=white, scale=0.9] at
            ($(brewer0)!1.5cm!(brewer0_ground)$) {Brewing...};
        \draw[->] ($(brewer0)!2.9cm!(brewer0_ground)$) --
            node[above, scale=0.9, midway]{Coffee ready}
            ($(barista)!2.9cm!(barista_ground)$);
        \draw[->] ($(barista)!3cm!(barista_ground)$) --
            node[above, scale=0.9, midway]{Serve $Order_0$}
            ($(customers)!3cm!(customers_ground)$);

        \node[draw=none, fill=white, scale=0.9] at
            ($(customers)!1.5cm!(customers_ground)$)
            {Next customer};

        \draw[<-] ($(barista)!2cm!(barista_ground)$) --
            node[above, scale=0.9, pos=0.4]{Place $Order_1$}
            ($(customers)!2cm!(customers_ground)$);
        \draw[<-, dashed] ($(dots)!2.1cm!(dots_ground)$) --
            node[above, scale=0.9, pos=0.65]{Start brewer}
            ($(barista)!2.1cm!(barista_ground)$);

        \node[draw=none, fill=white, scale=0.9] at
            ($(customers)!3.5cm!(customers_ground)$)
            {Next customer};

        \node[draw=none, fill=white, scale=0.9] at ($(brewern)!4cm!(brewern_ground)$)
            {...};
        \node[draw=none, fill=white, scale=0.9] at ($(brewer0)!4cm!(brewer0_ground)$)
            {...};
        \node[draw=none, fill=white, scale=0.9] at ($(barista)!4cm!(barista_ground)$)
            {...};
        \node[draw=none, fill=white, scale=0.9] at ($(customers)!4cm!(customers_ground)$)
            {...};

        \draw[<-] ($(barista)!4.5cm!(barista_ground)$) --
            node[above, scale=0.9, midway]{Place $Order_j$}
            ($(customers)!4.5cm!(customers_ground)$);
        \draw[<-] ($(brewern)!4.6cm!(brewern_ground)$) --
            node[above, scale=0.9, pos=0.75]{Start brewer}
            ($(barista)!4.6cm!(barista_ground)$);
        \node[draw=none, fill=white, scale=0.9] at
            ($(brewern)!5.5cm!(brewern_ground)$) {Brewing...};
        \draw[->] ($(brewern)!6.4cm!(brewern_ground)$) --
            node[above, scale=0.9, pos=0.75]{Coffee ready}
            ($(barista)!6.4cm!(barista_ground)$);
        \draw[->] ($(barista)!6.5cm!(barista_ground)$) --
            node[above, scale=0.9, midway]{Serve $Order_j$}
            ($(customers)!6.5cm!(customers_ground)$);

        \draw[->, dashed] ($(dots)!5.5cm!(dots_ground)$) --
            node[above, scale=0.9, pos=0.41]{Serve $Order_{j-1}$}
            ($(customers)!5.5cm!(customers_ground)$);

        \node[draw=none, fill=white, scale=0.9] at ($(brewern)!7cm!(brewern_ground)$)
            {...};
        \node[draw=none, fill=white, scale=0.9] at ($(brewer0)!7cm!(brewer0_ground)$)
            {...};
        \node[draw=none, fill=white, scale=0.9] at ($(barista)!7cm!(barista_ground)$)
            {...};
        \node[draw=none, fill=white, scale=0.9] at ($(customers)!7cm!(customers_ground)$)
            {...};
    \end{tikzpicture}
    \caption{A sequence diagram of a cooperative cafe order line. There is only
    one barista but $n$ brewers. As soon as a brewer is working, the next order
    can be handled.}
    \label{fig:cafe-cooperative}
\end{figure}

Looking at all three of these order handling approaches, we find a few common
entities. They all have orders, a customer line, one or several baristas and
brewers. Lets generalize this a bit and call orders \textit{events}, the
customer line we call the \textit{dispatcher}, the barista and brewer combined
we call the \textit{event handler} and the work being done on event
$\epsilon_j$ we call $\Lambda_j$. The definition of the three event propagation
models is as follows:

\begin{description}

    \item[\textbf{Serial event propagation}:] Given a set of $m$ events $\{
            \epsilon_0, \epsilon_1, ..., \epsilon_j, ..., \epsilon_{m-1} \}$ no
        event $\epsilon_j$ is dispatched before the work $\Lambda_{j-1}$ is
        finished, see Figure \ref{fig:event-propagation-serial}.

    \item[\textbf{Preemptive event propagation}:] Given a set of $m$ events $\{
            \epsilon_0, \epsilon_1, ..., \epsilon_j, ..., \epsilon_{m-1} \}$
        each event $\epsilon_j$ can be dispatched independent on
        $\Lambda_{j-1}$ and the work $\Lambda_j$ can implicitly interleave and
        make room for work $\Lambda_k$ where $k \ne j$.

    \item[\textbf{Cooperative event propagation}:] Given a set of $m$ events
        $\{ \epsilon_0, \epsilon_1, ..., \epsilon_j, ..., \epsilon_{m-1} \}$
        each event $\epsilon_j$ can be dispatched independent on
        $\Lambda_{j-1}$ and the work $\Lambda_j$ can explicitly make room for
        work $\Lambda_k$ where $k \ne j$.

\end{description}

\begin{figure}[h!]
    \centering
    \begin{tikzpicture}[node distance=5cm, auto, >=stealth']
        \node[] (event_handler) {Event handler};
        \node[right of=event_handler] (dispatcher) {Dispatcher};

        \node[below of=event_handler, node distance=8cm] (event_handler_ground) {};
        \node[below of=dispatcher, node distance=8cm] (dispatcher_ground) {};

        \draw (event_handler) -- (event_handler_ground);
        \draw (dispatcher) -- (dispatcher_ground);

        \draw[<-] ($(event_handler)!1cm!(event_handler_ground)$) --
            node[above, scale=0.9, midway]{Dispatch $\epsilon_0$}
            ($(dispatcher)!1cm!(dispatcher_ground)$);
        \node[draw=none, fill=white, scale=0.9] at
            ($(event_handler)!1.5cm!(event_handler_ground)$)
            {$\Lambda_0$};
        \draw[->] ($(event_handler)!2cm!(event_handler_ground)$) --
            node[above, scale=0.9, midway]{Finish $\epsilon_0$}
            ($(dispatcher)!2cm!(dispatcher_ground)$);

        \node[draw=none, fill=white, scale=0.9] at
            ($(dispatcher)!2.5cm!(dispatcher_ground)$)
            {Next event};

        \draw[<-] ($(event_handler)!3cm!(event_handler_ground)$) --
            node[above, scale=0.9, midway]{Dispatch $\epsilon_1$}
            ($(dispatcher)!3cm!(dispatcher_ground)$);
        \node[draw=none, fill=white, scale=0.9] at
            ($(event_handler)!3.5cm!(event_handler_ground)$)
            {$\Lambda_1$};
        \draw[->] ($(event_handler)!4cm!(event_handler_ground)$) --
            node[above, scale=0.9, midway]{Finish $\epsilon_1$}
            ($(dispatcher)!4cm!(dispatcher_ground)$);

        \node[draw=none, fill=white, scale=0.9] at
            ($(dispatcher)!4.5cm!(dispatcher_ground)$)
            {Next event};

        \node[draw=none, fill=white, scale=0.9] at
        ($(event_handler)!5cm!(event_handler_ground)$)
            {...};
        \node[draw=none, fill=white, scale=0.9] at
        ($(dispatcher)!5cm!(dispatcher_ground)$)
            {...};

        \draw[->] ($(event_handler)!5.5cm!(event_handler_ground)$) --
            node[above, scale=0.9, midway]{Finish $\epsilon_{j-1}$}
            ($(dispatcher)!5.5cm!(dispatcher_ground)$);

        \node[draw=none, fill=white, scale=0.9] at
            ($(dispatcher)!6cm!(dispatcher_ground)$)
            {Next event};

        \draw[<-] ($(event_handler)!6.5cm!(event_handler_ground)$) --
            node[above, scale=0.9, midway]{Dispatch $\epsilon_j$}
            ($(dispatcher)!6.5cm!(dispatcher_ground)$);
        \node[draw=none, fill=white, scale=0.9] at
            ($(event_handler)!7cm!(event_handler_ground)$)
            {$\Lambda_j$};
        \draw[->] ($(event_handler)!7.5cm!(event_handler_ground)$) --
            node[above, scale=0.9, midway]{Finish $\epsilon_j$}
            ($(dispatcher)!7.5cm!(dispatcher_ground)$);

        \node[draw=none, fill=white, scale=0.9] at
        ($(event_handler)!8cm!(event_handler_ground)$)
            {...};
        \node[draw=none, fill=white, scale=0.9] at
        ($(dispatcher)!8cm!(dispatcher_ground)$)
            {...};
    \end{tikzpicture}
    \caption{The serial event propagation model. Only one event $\epsilon_j$ is
    dispatched and handled at a time and no event $\epsilon_j$ can be
    dispatched before the previous work $\Lambda_{j-1}$ is finished.}
    \label{fig:event-propagation-serial}
\end{figure}

\subsection{The anatomy of an event}

The event is the actual data transmitted from the device to the gateway.
Imagine a small weather station used at private homes. It has two sensors
measuring wind velocity and air temperature. The gateway is the central system
passing the data from each sensor to the application using the data. At a
certain frequency, the gateway polls each sensor for new velocity and
temperature values. These values are the actual events in the system. In this
toy example, each event might not only carry the actual value of the sensor it
corresponds to, but also information on what sensor this data is coming from.
The weather application might draw a plot on how the air temperate has changed
over time, therefore each event must also include the timestamp of each sensor
value. In this study, however, there is no need for event data that represents
some mimicked sensor value. What is needed is a way to track each event to see
when it reaches certain places in its life. This is done by registering the
timestamp when the event reaches those places, see Figure \ref{fig:event-life}.

\begin{figure}[h!]
    \centering
    \begin{tikzpicture}[node distance=3cm, auto]
        \node[block] (device) {Device};
        \node[block, below of=device] (dispatcher) {Dispatcher};
        \node[block, below of=dispatcher] (event_handler) {Event Handler};

        \path[line] (device) -- (dispatcher);
        \path[line] (dispatcher) -- (event_handler);

        \node[comment, node distance=1cm, right =of device] (created) {Created};
        \node[comment, node distance=1cm, right =of dispatcher] (dispatched) {Dispatched};
        \node[comment, node distance=1cm, right =of event_handler] (processed) {Processed};
    \end{tikzpicture}
    \caption{The event travels from the device down to the event handler. At
    each step the time point is registered, shown in the clouds.}
    \label{fig:event-life}
\end{figure}

\section{An abstract gateway}

A model of an abstract gateway is proposed as such: a gateway $\Gamma$ is a
five-tuple describing its internal properties and its environment.

\begin{equation}
\Gamma = \big \langle \Omega, \phi, \delta, K, \Lambda \big \rangle
\end{equation}

Let $\Omega$, $\phi$ and $\delta$ describe the properties of the environment of
$\Gamma$. $\Omega$ describes the devices communicating with $\Gamma$ and can be
expressed as a set $\Omega = \{ \omega_0, \omega_1, ..., \omega_{q-1} \}$ where
$\omega_i$ represents a device and $q$ is the total number of devices. Let
$\phi$ be the number of events $\omega_i$ generates each second and let
$\delta$ be the time it takes for a network request to travel from $\omega_i$
to $\Gamma$, or vice versa. $K$ and $\Lambda$ describes the internal properties
of $\Gamma$. Let $K$ describe the event propagation model of $\Gamma$. It is
expressed as $K = \{ \kappa_0, \kappa_1 \}$ where $\kappa_0$ and $\kappa_1$
represents the event propagation model of the dispatcher and the event handler
respectively and $\kappa_i \in \{ \textit{serial}, \textit{preemptive},
\textit{cooperative} \}$. Let $\Lambda$ describe the resource demand in
$\Gamma$ from each event. It is expressed as $\Lambda = \{ \lambda_0, \lambda_1
\}$ where $\lambda_0$ and $\lambda_1$ represents the CPU intensity and the I/O
intensity induced by each event, where $0 \leq \lambda_i \leq 1$.

In addition to the rather static expression of an abstract gateway $\Gamma$,
the function $E$ is provided:

\begin{equation}
E(\Gamma, t_{start}, t_{end}) \to M = \big[ \epsilon_0, \epsilon_1, ...,
\epsilon_{m-1} \big]^T
\end{equation}

By providing $E$ with $\Gamma$ and two points in time $t_{start}$ and $t_{end}$
where $t_{start} \leq t_{end}$, a vector $M$ of ordered events $\epsilon_j$ is
returned, where $\epsilon_j = [ t_0^j, t_1^j, t_2^j, t_3^j, t_4^j ]$. Note that
$M$ is ordered by $t_0^j$ and $m$ is the total number of events. Each
$\epsilon_j$ include five timestamps $t_k^j$ that represents parts of the event
lifecycle:

\begin{description}
\item[$t_0^j$] represents the point in time when $\epsilon_j$ was created in
    $\omega_i$.
\item[$t_1^j$] represents the point in time when $\epsilon_j$ left $\omega_i$.
\item[$t_2^j$] represents the point in time when $\epsilon_j$ arrived at
    $\Gamma$.
\item[$t_3^j$] represents the point in time when $\epsilon_j$ was dispatched to
    the event handler in $\Gamma$.
\item[$t_4^j$] represents the point in time when $\Gamma$ finished processing
    $\epsilon_j$.
\end{description}

These timestamps holds a set of basic constraints:

\begin{equation}
t_{start} \leq t_k^j
\end{equation}

\begin{equation}
t_0^j \leq t_{end}
\end{equation}

\begin{equation}
t_0^j \leq t_1^j \leq t_2^j \leq t_3^j \leq t_4^j
\end{equation}

By taking $\delta$ into account, we

\todo{A sequence diagram visualizing the event lifecycle?}

\subsection{Analysing the performance}

With $M$ and $\epsilon_j$ it is possible to derive some performance metrics.
Throughput $T$ can be calculated as:

$$
T = \frac{\sum_{\epsilon_j \in M}{\pi(\epsilon_j)}}{t_{end} - t_{start}}
\text{ [events / s]}
$$

It is also possible to analyze the life length of an event, how long time it
spent in $\omega_i$ and how long time it spent in $\Gamma$. Let:

\begin{description}
    \item[$d_0^j = t_4^j - t_0^j$] be the life length of $\epsilon_j$, i.e. the
        total time between that the event was created until it was processed.
    \item[$d_1^j = t_1^j - t_0^j$] be the time $\epsilon_j$ spent in
        $\omega_i$.
    \item[$d_2^j = t_4^j - t_2^j$] be the time $\epsilon_j$ spent in
        $\Gamma$.
\end{description}

\subsection{Analysing the load}

A system is said to be \textit{at load} when the system is working at full
capacity and is not able to perform any better \todo{Reference?}. Due to the
strict time limitation provided by $t_{end}$ there might be cases where some
events $\epsilon_j$ has $t_k^j \geq t_{end}$. These are events that haven't
been entirely processed by $\Gamma$. Let $\pi$ be a function such that:

\[
    \pi(\epsilon_j) =
\begin{cases}
    1, & \text{if } t_4^j \leq t_{end} \\
    0, & \text{otherwise}
\end{cases}
\]

With $\pi$, the load on $\Gamma$ can be analysed. The ratio between created
events and processed events can give hint on how many events $\Gamma$ can
handle. Let $P$ denote the ratio between created and processed events:

$$
P = \frac{m}{P_\pi} = \frac{m}{\sum_{\epsilon_j \in M}{\pi(\epsilon_j)}}
$$

$P$ will always be in the interval $0 \leq P \leq 1$. In cases where $P \simeq
1$, $\Gamma$ is able to process most of the events and it might even be so that
$\Gamma$ is waiting for new events to be created. In that case, we can not be
sure whether $\Gamma$ is at a proper load or not. However, in cases where $P
\ll 1$ we know for sure $\Gamma$ is not able to process all events and is not
waiting for new events to be created. A proper load is therefore one where
$P_\pi \ll m$. $m$ can be calculated as:

$$
m = q \times \phi \times (t_{end} - t_{start})
$$

The variable affecting $P_\pi$ is basically just the condition $t_4^j \leq
t_{end}$. Increasing $t_{end}$ is not a solution since that will also increase
$m$. We know that $t_4^j \geq t_$

\section{Development methodology}

This entire study will be conducted with Scrum as its backbone. Both the
writing of the thesis and the development of the applications to test will
happen in sprints. The project's backlog will initially be the fundamental
requirements of a master's thesis (based on requirements from Linköping
University) and the research questions. The forms of its user stories will
resemble traditional issues or requirements, but their scope can be wide and
their acceptance criteria abstract. Every week they undergo refinement and
abstract stories will be split into concrete ones as new knowledge about them
is acquired during the project. For instance, a starting user story (or issue)
will be \textit{"write the Results chapter"}. Initially this story is very
large, it is hard to do it in one sprint and it is hard to know where to start.
As time progress and the application to test have been developed and tests have
been conducted, the story can be split into more precise issues like
\textit{"present the developed application"} and \textit{"present a diagram
with the test results"}. These issues are (subjectively) easier to do in one
sprint and it is also easier to know when they are finished.

The author will take on all three traditional Scrum roles: product owner, Scrum
master and team member. Other stakeholders of the project are representatives
from Attentec, the examiner and a mentor from Linköping University. Each sprint
will have a length of 2 weeks and the sprint planning will, unlike the
traditional sprint planning \cite{sims2012scrum} where only the product owner
and team members are present, include a mentor from Attentec to help plan the
upcoming sprint. At the end of each sprint the current status of the project
will be presented to the stakeholders.

\section{Literature study}

A major literature study will be conducted prior and during the implementation.
The theoretical framework will be vindicated in this phase to support claims
and form a general direction of the entire study. Multiple databases will be
queried in order to find interesting material from previous research. Mainly
the online library hosted by \textit{Linköping
University}\footnote{\url{www.bibl.liu.se}} will be used since it allow access
to material otherwise unviewable due to institutional login requirements. Query
results from this library is a collection of query results from other research
databases such as \textit{ACM Digital Library}\footnote{\url{dl.acm.org}},
\textit{ProQuest Ebook Central}\footnote{\url{ebookcentral.proquest.com}} and
\textit{IEEE Xplore Digital Library}\footnote{\url{ieeexplore.ieee.org}}; so it
acts as a gateway to a global collection of scientific research.

The \textit{three-pass approach} presented by Keshav \cite{keshav2007read} will
be used as a basic approach to find interesting material. It helps the reader
grasp the paper's content in three \textit{passes}. The first pass' purpose is
to give the reader an overview of the paper. The title, abstract, introduction,
headings, sub-headings, conclusions and references are read. This information
should help the reader understand the paper's category and context and help
decide whether to continue read this paper or leave it. If the reader choose to
continue read it, the second pass starts. Here the paper is read more
thoroughly. The figures and diagrams are examined and after this pass the
reader should be able to summarize the paper, with leading evidence, to someone
else. The purpose of the third pass is to fully understand the paper. By making
the same assumptions as the author, the paper is virtually re-created. It helps
identify the true innovations of the paper, as well as the hidden failures.

\section{Implementation}

The main approach this study will undertake to answer its question is to:

\begin{enumerate}
  \item find a state of the art reactive system in the industry with an
    appropiate level of complexity
  \item create a specification of the system
  \item reimplement the system with libuv
  \item create and conduct tests
  \item apply and evaluate the maintainability metrics on both systems
  \item present the results
\end{enumerate}

With help from Attentec, a multi-sensor monitor application used by one of
their clients will be specified and its source code will be used to compare its
software maintainability to a reimplementation with libuv. If no appropiate
system can be found in the industry, Attentec will aid in creating a
specification for a similar system as well as a state of the art technique to
implement it. The specification will follow the same pattern presented by Ardis
et al \cite{ardis1996framework}.

Even though libuv is written in C, C++ will be used as the main programming
language for the libuv implementation. With the dynamic programming style C++
offers with classes and templates, it will be more suitable for an
implementation that hopefully will attract web developers.

A testing environment will be created to simulate an IoT context where multiple
sensors are connected to the monitor application. A software testing suite will
also be setup to create unit tests.

\section{Evaluation}

Same software maintainability metrics will be applied on the system found in
the industry and the reimplementation of it. The results of the metrics will be
compared and presented.
