\usepackage[backend=biber,hyperref]{biblatex}
%% To set the font of your thesis, use the \setmainfont{} command,
%% surrounded with \ifxetex if you want to switch between xelatex and pdflatex
\ifxetex
%\setmainfont [Scale=1]{Georgia}
\fi

%%%%%%%%%%%%
%% The VZ43 chapter style, from Memoir contributed chapter styles: ftp://ftp.tex.ac.uk/ctan%3A/info/MemoirChapStyles/MemoirChapStyles.pdf
%%%%%%%%%%%

\usepackage{calc,color}
\newif\ifNoChapNumber
\newcommand\Vlines{%
\def\VL{\rule[-2cm]{1pt}{5cm}\hspace{1mm}\relax}
\VL\VL\VL\VL\VL\VL\VL}
\makeatletter
\setlength\midchapskip{0pt}
\makechapterstyle{VZ43}{
\renewcommand\chapternamenum{}
\renewcommand\printchaptername{}
\renewcommand\printchapternum{}

\renewcommand\chapnumfont{\Huge\bfseries\centering}
\renewcommand\chaptitlefont{\Huge\bfseries\raggedright}
\renewcommand\printchaptertitle[1]{%
\Vlines\hspace*{-2em}%
\begin{tabular}{@{}p{1cm} p{\textwidth-3cm}}%
\ifNoChapNumber\relax\else%
\colorbox{black}{\color{white}%
\makebox[.8cm]{\chapnumfont\strut \thechapter}}
\fi
& \chaptitlefont ##1
\end{tabular}
\NoChapNumberfalse
}
\renewcommand\printchapternonum{\NoChapNumbertrue}
}
\makeatother


%% To set bibliography options, refer to the biblatex manual and use
%% the ExecuteBibliographyOptions command below to set your options

\ExecuteBibliographyOptions{maxnames=99}
\ExecuteBibliographyOptions{sorting=none}


%% Change this to your appropriate BibTeX reference file (.bib)

\addbibresource{references.bib}

%% Listing settings

\usepackage{listings}

\lstset{
  breaklines=true,
  xleftmargin=\parindent,
  belowcaptionskip=1\baselineskip,
  belowskip=1\baselineskip,
  basicstyle=\footnotesize\ttfamily,
  keywordstyle=\color{blue!40!black},
  commentstyle=\color{green!40!black},
  identifierstyle=\color{black},
  stringstyle=\color{orange},
  showstringspaces=false,
}

\lstdefinelanguage{javascript}{
  morekeywords={typeof, new, true, false, catch, function, return, null, catch,
  switch, var, const, let, if, in, while, do, else, case, break},
  morecomment=[s]{/*}{*/},
  morecomment=[l]//,
  morestring=[b]",
  morestring=[b]'
}

%% Sectional labeling level.

\setcounter{secnumdepth}{2}

%% Centered table

\newenvironment{ctable}[2] {
\begin{table}[htbp]
  \caption{#1}
  \begin{center}
    \begin{tabular}{#2}
}
{
    \end{tabular}
  \end{center}
\end{table}
}

%% Tikz lib

\usepackage{tikz}
\usepackage{pgfplots}
\usepackage{pgfplotstable}
\usepackage{tikzscale}
\usetikzlibrary{shapes, arrows, calc, positioning, fit, matrix, er}
\usepgfplotslibrary{groupplots}

\tikzstyle{block} = [
    rectangle,
    draw,
    fill = blue!5,
    text centered,
    rounded corners,
    inner sep = 0.5cm,
    outer sep = 0.1cm
]

\tikzstyle{decision} = [
    diamond,
    draw,
    fill = blue!5,
    text badly centered,
    node distance = 2.5cm,
    inner sep = 0.1cm,
    outer sep = 0.1cm
]

\tikzstyle{line} = [draw, -latex']

\tikzstyle{comment} = [
    cloud,
    fill = yellow!15,
    inner sep = 0.3cm,
    cloud ignores aspect
]

\tikzstyle{fitblock} = [
    draw = black,
    fill = white,
    rounded corners,
    text width = 2.5cm,
    font = {\sffamily\bfseries\color{black}},
    align = center,
    text height = 12pt,
    text depth = 9pt
]

\tikzstyle{layeredblocks} = [
    matrix of nodes,
    nodes in empty cells,
    nodes = {fitblock},
    row sep = 3pt,
    column sep = 3pt
]

\tikzstyle{classblock} = [
    block,
    inner sep = 0.2cm,
    text width = 4cm,
    rectangle split,
    rectangle split parts = 2,
    every second node part/.style={align = left},
    every third node part/.style={align = left},
]

\pgfkeys{
    /tikz/node distance/.append code = {
        \pgfkeyssetvalue{/tikz/node distance value}{#1}
    }
}

\newcommand{\spanblock}[3]{
    \node[
        fitblock,
        inner sep = 0pt,
        fit = {(#1) (#2)},
        label = center:{\sffamily\bfseries\color{black}#3}
    ] {};
}

\newcommand{\performanceplot}[2]
{
    \begin{tikzpicture}
        \pgfplotstableread[col sep = semicolon]{#1}\csvdata
        \begin{axis}[#2]
            \addplot table[y=serial/serial]{\csvdata};
            \addlegendentry{s/s}
            \addplot table[y=serial/preemptive]{\csvdata};
            \addlegendentry{s/p}
            \addplot table[y=cooperative/cooperative]{\csvdata};
            \addlegendentry{c/c}
            \addplot table[y=cooperative/preemptive]{\csvdata};
            \addlegendentry{c/p}
        \end{axis}
    \end{tikzpicture}
}

\newcommand{\nextperformanceplot}[2]
{
    \nextgroupplot[#2]
    \addplot table[y = serial/serial, col sep = semicolon]{#1};
    \addplot table[y = serial/preemptive, col sep = semicolon]{#1};
    \addplot table[y = cooperative/cooperative, col sep = semicolon]{#1};
    \addplot table[y = cooperative/preemptive, col sep = semicolon]{#1};
}

\newcommand{\nextperformanceplotwithlegend}[2]
{
    \nextgroupplot[#2]
    \addplot table[y = serial/serial, col sep = semicolon]{#1};
    \addlegendentry{s/s}
    \addplot table[y = serial/preemptive, col sep = semicolon]{#1};
    \addlegendentry{s/p}
    \addplot table[y = cooperative/cooperative, col sep = semicolon]{#1};
    \addlegendentry{c/c}
    \addplot table[y = cooperative/preemptive, col sep = semicolon]{#1};
    \addlegendentry{c/p}
}

%% enumitem

\usepackage{enumitem}

\setlist[description]{leftmargin=\parindent,labelindent=\parindent}
