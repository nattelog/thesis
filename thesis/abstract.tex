This study focuses on the Internet of Things (IoT) gateway; a common middleware
solution that bridges the gap between physical sensors and devices to internet
applications. There is a shown interest in understanding the characteristics of
different types of gateway architectures both from the research field and the
industry, particularily the IT-consulting firm Attentec in Linköping, Sweden.
Built on top of task management theory, a set of three general architectural
approaches are proposed. Also, two primary components called the dispatcher and
the event handler are proposed as the primary components of \textit{any}
gateway implementation. Common internal and external properties are identified
based on state-of-the-art gateway implementations found in the industry and
professional guidance. All of these mentioned variables are taken into account
when a general gateway implementation is developed that is proposed to be
configurable to mimick any architectural level implementation of the gateway. A
set of performance tests are conducted on the implementation to observe how
different configurations of the gateway affect throughput and response time of
data transmitted from simulated devices.

Apart from this general approach a study has also been made on the open source
C library \textit{libuv}, used in the common web runtime engine
\textit{NodeJS}. The library has been used to study how asynchronous I/O
operations can be used to improve the IoT gateway performance.
