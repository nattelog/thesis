This study focuses on the Internet of Things (IoT) gateway; a common middleware
solution that bridges the gap between physical sensors and devices to internet
applications. There is a shown interest in understanding the characteristics of
different types of gateway architectures both from the research field and the
industry, particularly the IT-consulting firm Attentec in Linköping, Sweden.  A
study has also been made on the open source C library \textit{libuv}, used in
the common web runtime engine \textit{NodeJS}. The library has been used to
study how asynchronous I/O operations can be used to improve the IoT gateway
performance. A set of three general architectural approaches are identified.
Common internal and external properties are identified based on
state-of-the-art gateway implementations found in the industry. All of these
properties are taken into account when a general gateway implementation is
developed that is proposed to mimic any architectural level implementation of
the gateway. A set of performance tests are conducted on the implementation to
observe how different configurations of the gateway affect throughput and
response time of data transmitted from simulated devices. The results show that
the properties of the gateway do affect throughput and response time
significantly and that libuv overall helps implement one of the best performing
gateway configurations.

