\chapter{Introduction}
\label{cha:introduction}

The Internet of Things (IoT) is a new dimension of the internet which includes
\textit{smart devices} - physical devices able to communicate with the outside
world through internet
\cite{design-principles-for-distributed-embedded-applications}. Ericsson
foretold there will be around 400 million smart devices by the end of 2016
\cite{ericsson-iot-forecast} and a great challenge will be to serve all of
these. IoT applications are already influencing our everyday lives; smart
cities, smart grids and traffic management are only a few of the large variety
of applications in this "multi-tiered heterogeneous system based on open
architectural platforms and standards" \cite{gardavsevic2017iot}. The
complexity of a smart device varies from passive ID-tags which communicates
through near field communication to full scale computers with multithreaded
operating systems.

Advanced RISC Machine (ARM) processors are used in different types of
electronic devices. RISC is short for \textit{Reduced Instruction Set Computer}
and its supported instructions are simple, have uniform length and almost all
instructions execute in one clock cycle. This reduces the complexity of the
chip and gives more room to performance enhancing features \cite{risc-vs-cisc}.
In an IoT context, ARM processors can be a good choice to be embedded on a
smart device due to its large variety in cost, energy consumption, performance
and size.

From a network perspective, smart devices can be seen as clients in a
traditional client-server architecture. They emit data to a server through an
internet protocol. The server receives the data and processes it. When building
these kinds of network services it is important for the server to avoid
blocking concurrent requests \cite{elmeleegy2004lazy}. These requests are both
on a network level but also on the general operating system level. Requests
from a smart device might be logged on a file or written to a database, meaning
the server must start new processes to handle these requests without blocking
new requests from the network. A server does not necessarily have to be a web
server serving requests on the internet protocol. The operating system itself
can be viewed as an I/O (input/output) server receiving requests from the I/O
layer. This can be data on serial ports and general purpose I/O pins on the
processor. A device monitoring a physical environment using sensors can take
use of this feature by letting the registers call interrupt handlers when they
are ready for reading or writing. Thus an application running on an embedded
system listening for data from sensors can be viewed as a web server listening
for internet requests. The same challenge regarding non-blocking request
handlers holds also for these mentioned applications.

To implement a non-blocking server, the \textit{libuv} C++ library can be a
good choice. With it the user can implement an \textit{event driven}
architecture with request handlers reacting to certain events in the system,
e.g. I/O events from the operating system. \cite{libuv-webpage}

To determine the maintainability of a software system one can, among others,
look at the source code and its age since release, the number of non-commented
source code statements and its rate of failures per time unit.
\cite{oman1992metrics}

\section{Motivation}
\label{sec:motivation}

Since IoT is growing and communication will largely increase, efficiency and
performance is an important factor. There exists a number of communication
protocols for IoT network architectures. A big challenge is to support massive
data streams with minimal overhead in transport. However, this implies for the
transport and communication perspective on IoT, but another significant
perspective is I/O (file writes, database queries) processing on the machines
themselves. If the demand for high performance IoT services is increasing, then
development towards embedded systems will too. A rigor evaluation on two common
architectures can therefore be of great value to find what suits best in IoT.

\textit{libuv} is the major subsystem to \textit{Node.js}, a popular universal
language for \textit{"[...] front end, back end and connected devices [and]
everything from the browser to your toaster [...]."}
\cite{node-js-survey-report-2016}. Due to \textit{Node.js}' event-driven
development style many developers in its field might have a better experience
using \textit{libuv} when if need to transfer to embedded programming will be
in question.

\section{Aim}
\label{sec:aim}

To conclude whether \textit{libuv} is a good option in terms of performance and
maintainability when selecting underlying architecture for IoT devices.

\section{Research questions}
\label{sec:research-questions}

\begin{enumerate}
    \item How does an event-driven architecture using libuv compare to a
        multi-threaded environment in terms of performance?

    \item How does an event-driven architecture using libuv compare to a
        multi-threaded environment in terms of maintainability?
\end{enumerate}

\section{Delimitations}
\label{sec:delimitations}

Maintainability will be the main metric used. Other metrics might be useful for
evaluation of the architectures, but due to time limitations only
maintainability will be used. Only one or two types of devices will be used for
implementation and testing. This might decrease data resolution.
