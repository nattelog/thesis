\chapter{Introduction}
\label{cha:introduction}

The Internet of Things (IoT) is a new dimension of the internet which includes
\textit{smart devices} - physical devices able to communicate with the outside
world through internet
\cite{design-principles-for-distributed-embedded-applications}
\cite{wu2011m2m}. Ericsson foretold there would be around 400 million smart
devices by the end of 2016 \cite{ericsson-iot-forecast} and a great challenge
will be to serve all of these. IoT applications are already influencing our
everyday lives; smart cities, smart grids and traffic management are only a few
of the large variety of applications in this "multi-tiered heterogeneous system
based on open architectural platforms and standards" \cite{gardavsevic2017iot}.
The complexity of a smart device varies from passive ID-tags which communicate
through near field communication to full scale computers with multithreaded
operating systems.

An important component in the IoT architecture is the IoT gateway
\cite{chen2011brief}. It acts as a bridge between the "things", or devices, and
the internet. Gateways can be seen as \textit{event-driven}, or
\textit{reactive} systems, i.e. systems required to respond to stimuli in their
environment \cite{garlan1993introduction} \cite{harel1985development}.
Specifying reactive systems can be done by designing them as \textit{state
machines} and implementation can be done in \textit{reactive languages}
\cite{bainomugisha2013survey}. These languages have abstracted away time
management, in the same sense memory management is abstracted away in, e.g.
Java, by using garbage collectors.

This thesis has been conducted at the Linköping-based company Attentec in
Sweden. They focus on IoT-development and the gateways used by some of their
customers are implemented on embedded, resource-constrained hardware.  A common
language for embedded systems is C as it allows the developer to access
hardware without losing the benefits of high-level programming
\cite{nahas2012choosing} \cite{barr1999programming}. The C library
\textit{libuv} can be a good choice when implementing reactive systems in C
\cite{libuv-webpage}. With it the user can implement an \textit{event driven},
\textit{reactive} architecture with request handlers reacting to certain events
in the system, e.g. I/O events from the operating system. This work examines
the performance characteristics of the IoT gateway when its internal and
external properties are varied and verifies if libuv is a good choice of
approach when implementing an IoT gateway in C.

\section{Motivation}
\label{sec:motivation}

Earlier work has attempted to quantify the performance characteristics of
gateways, but most focus has been on either higher level architectural models
\cite{chen2011brief} \cite{zachariah2015internet} or hardware performance
characteristics \cite{kruger2014benchmarking}. Observing the performance while
varying software-related properties of the gateway will therefore be a good
contribution to the research-field. Since IoT and gateway technology is growing
and as the demand for its services will largely increase \cite{wu2011m2m},
efficiency and performance is an important factor. There exist a number of
communication protocols for IoT network architectures. A big challenge is to
support massive data streams with minimal overhead in transport. However, this
applies to the transport and communication perspective on IoT, but another
significant perspective is I/O (file writes, database queries) processing on
the machines themselves. If the demand for high performance IoT services is
increasing, then development towards embedded systems will too. A rigorous
evaluation of the different architectures a gateway can be developed on can
therefore be of great value to find what suits best in IoT.

\textit{NodeJS}, a popular universal runtime platform for \textit{"front end,
back end and connected devices [and] everything from the browser to your
toaster"}, includes the major subsystem libuv
\cite{node-js-survey-report-2016}. Due to NodeJS' event-driven development
style many developers in its field might have a better experience using libuv
if transfer to embedded programming could be in question.

\section{Aim}
\label{sec:aim}

The aim of this thesis is to understand and map the internal and external
properties of the gateway to its functionality and performance and to assess
the legitimacy of using libuv as an implementation approach when developing a
new IoT gateway.

\section{Research questions}
\label{sec:research-questions}

\begin{enumerate}

    \item What internal and external properties affect the functionality and
        performance of the IoT gateway?

    \item How can libuv be used in order to implement an IoT gateway? What are
        the benefits and disadvantages of doing so?

\end{enumerate}

\section{Delimitations}
\label{sec:delimitations}

Performance will be the only attribute used to determine whether a gateway
implementation is "good" or not. There are other ways of measuring programs as
well, one of them are code measurements such as maintainability. However, this
study only focuses on performance measures.

The context of the study is IoT gateways. This study can however be applicable
on a larger spectrum of environments, e.g. web servers or enterprise systems.
