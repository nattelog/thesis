\documentclass{beamer}

\usepackage[backend=biber,hyperref]{biblatex}
%% To set the font of your thesis, use the \setmainfont{} command,
%% surrounded with \ifxetex if you want to switch between xelatex and pdflatex
\ifxetex
%\setmainfont [Scale=1]{Georgia}
\fi

%%%%%%%%%%%%
%% The VZ43 chapter style, from Memoir contributed chapter styles: ftp://ftp.tex.ac.uk/ctan%3A/info/MemoirChapStyles/MemoirChapStyles.pdf
%%%%%%%%%%%

\usepackage{calc,color}
\newif\ifNoChapNumber
\newcommand\Vlines{%
\def\VL{\rule[-2cm]{1pt}{5cm}\hspace{1mm}\relax}
\VL\VL\VL\VL\VL\VL\VL}
\makeatletter
\setlength\midchapskip{0pt}
\makechapterstyle{VZ43}{
\renewcommand\chapternamenum{}
\renewcommand\printchaptername{}
\renewcommand\printchapternum{}

\renewcommand\chapnumfont{\Huge\bfseries\centering}
\renewcommand\chaptitlefont{\Huge\bfseries\raggedright}
\renewcommand\printchaptertitle[1]{%
\Vlines\hspace*{-2em}%
\begin{tabular}{@{}p{1cm} p{\textwidth-3cm}}%
\ifNoChapNumber\relax\else%
\colorbox{black}{\color{white}%
\makebox[.8cm]{\chapnumfont\strut \thechapter}}
\fi
& \chaptitlefont ##1
\end{tabular}
\NoChapNumberfalse
}
\renewcommand\printchapternonum{\NoChapNumbertrue}
}
\makeatother


%% To set bibliography options, refer to the biblatex manual and use
%% the ExecuteBibliographyOptions command below to set your options

\ExecuteBibliographyOptions{maxnames=99}
\ExecuteBibliographyOptions{sorting=none}


%% Change this to your appropriate BibTeX reference file (.bib)

\addbibresource{references.bib}

%% Listing settings

\usepackage{listings}

\lstset{
  breaklines=true,
  xleftmargin=\parindent,
  belowcaptionskip=1\baselineskip,
  belowskip=1\baselineskip,
  basicstyle=\footnotesize\ttfamily,
  keywordstyle=\color{blue!40!black},
  commentstyle=\color{green!40!black},
  identifierstyle=\color{black},
  stringstyle=\color{orange},
  showstringspaces=false,
}

\lstdefinelanguage{javascript}{
  morekeywords={typeof, new, true, false, catch, function, return, null, catch,
  switch, var, const, let, if, in, while, do, else, case, break},
  morecomment=[s]{/*}{*/},
  morecomment=[l]//,
  morestring=[b]",
  morestring=[b]'
}

%% Sectional labeling level.

\setcounter{secnumdepth}{2}

%% Centered table

\newenvironment{ctable}[2] {
\begin{table}[htbp]
  \caption{#1}
  \begin{center}
    \begin{tabular}{#2}
}
{
    \end{tabular}
  \end{center}
\end{table}
}

%% Tikz lib

\usepackage{tikz}
\usepackage{pgfplots}
\usepackage{pgfplotstable}
\usepackage{tikzscale}
\usetikzlibrary{shapes, arrows, calc, positioning, fit, matrix}
\usepgfplotslibrary{groupplots}

\tikzstyle{block} = [
    rectangle,
    draw,
    fill = blue!5,
    text centered,
    rounded corners,
    inner sep = 0.5cm,
    outer sep = 0.1cm
]

\tikzstyle{decision} = [
    diamond,
    draw,
    fill = blue!5,
    text badly centered,
    node distance = 2.5cm,
    inner sep = 0.1cm,
    outer sep = 0.1cm
]

\tikzstyle{line} = [draw, -latex']

\tikzstyle{comment} = [
    cloud,
    fill = yellow!15,
    inner sep = 0.3cm,
    cloud ignores aspect
]

\tikzstyle{fitblock} = [
    draw = black,
    fill = white,
    rounded corners,
    text width = 2.5cm,
    font = {\sffamily\bfseries\color{black}},
    align = center,
    text height = 12pt,
    text depth = 9pt
]

\tikzstyle{layeredblocks} = [
    matrix of nodes,
    nodes in empty cells,
    nodes = {fitblock},
    row sep = 3pt,
    column sep = 3pt
]

\tikzstyle{classblock} = [
    block,
    inner sep = 0.2cm,
    text width = 4cm,
    rectangle split,
    rectangle split parts = 2,
    every second node part/.style={align = left},
    every third node part/.style={align = left},
]

\pgfkeys{
    /tikz/node distance/.append code = {
        \pgfkeyssetvalue{/tikz/node distance value}{#1}
    }
}

\newcommand{\spanblock}[3]{
    \node[
        fitblock,
        inner sep = 0pt,
        fit = {(#1) (#2)},
        label = center:{\sffamily\bfseries\color{black}#3}
    ] {};
}

\newcommand{\performanceplot}[2]
{
    \begin{tikzpicture}
        \pgfplotstableread[col sep = semicolon]{#1}\csvdata
        \begin{axis}[#2]
            \addplot table[y=serial/serial]{\csvdata};
            \addlegendentry{s/s}
            \addplot table[y=serial/preemptive]{\csvdata};
            \addlegendentry{s/p}
            \addplot table[y=cooperative/cooperative]{\csvdata};
            \addlegendentry{c/c}
            \addplot table[y=cooperative/preemptive]{\csvdata};
            \addlegendentry{c/p}
        \end{axis}
    \end{tikzpicture}
}

\newcommand{\nextperformanceplot}[2]
{
    \nextgroupplot[#2]
    \addplot table[y = serial/serial, col sep = semicolon]{#1};
    \addplot table[y = serial/preemptive, col sep = semicolon]{#1};
    \addplot table[y = cooperative/cooperative, col sep = semicolon]{#1};
    \addplot table[y = cooperative/preemptive, col sep = semicolon]{#1};
}

\newcommand{\nextperformanceplotwithlegend}[2]
{
    \nextgroupplot[#2]
    \addplot table[y = serial/serial, col sep = semicolon]{#1};
    \addlegendentry{s/s}
    \addplot table[y = serial/preemptive, col sep = semicolon]{#1};
    \addlegendentry{s/p}
    \addplot table[y = cooperative/cooperative, col sep = semicolon]{#1};
    \addlegendentry{c/c}
    \addplot table[y = cooperative/preemptive, col sep = semicolon]{#1};
    \addlegendentry{c/p}
}

%% enumitem

\usepackage{enumitem}

\setlist[description]{leftmargin=\parindent,labelindent=\parindent}


\mode<presentation>
{
    \usetheme{default}
    \usecolortheme{default}
    \usefonttheme{default}
    \setbeamertemplate{navigation symbols}{}
    \setbeamertemplate{caption}[numbered]
}

\usepackage[english]{babel}
\usepackage[utf8x]{inputenc}
\usepackage{graphics}

\title{A study on the performance and architectural characteristics of the
Internet of Things gateway}
\author{Natanael Log}
\institute{Master's thesis, spring 2018}
\date{2018-06-01}

\begin{document}
    \begin{frame}
        \titlepage
    \end{frame}

    \begin{frame}{Presentation outline}
        \tableofcontents
    \end{frame}

    \section{Background}

    \subsection{Internet of Things}

    \begin{frame}{Internet of Things}

        Things connected to the internet.

    \end{frame}

%    \begin{frame}{Internet of Things}
%
%        \begin{center}
%
%        \textit{"Internet of Things links uniquely identifiable things to their
%            virtual representations in the Internet [...]. The provided
%            accurate and appropriate information may be accessed in the right
%            quantity and condition, at the right time and place at the right
%            price. The Internet of Things is not synonymous with ubiquitous /
%            pervasive computing, the Internet Protocol (IP), communication
%            technology, embedded devices, its applications, the Internet of
%            People or the Intranet / Extranet of Things, yet it combines
%            aspects and technologies of all of these approaches."}
%        \footcite{uckelmann2011architectural}
%        \end{center}
%
%    \end{frame}

    \begin{frame}{IoT architecture}

        Chen et al. divides the IoT architecture into three layered domains
        \footcite{chen2011brief}:

        \begin{figure}[h!]
            \centering
            \begin{tikzpicture}
                \matrix (table) [widelayeredblocks] {
                    Application domain \\
                    Network domain \\
                    Sensing domain \\
                };
            \end{tikzpicture}
        \end{figure}

    \end{frame}

%    \begin{frame}{An IoT device example}
%
%        \begin{block}{Raspberry Pi\footnote{\url{raspberrypi.org}}}
%            \begin{itemize}
%                \item Can talk directly to the internet
%                \item Relatively powerful\footcite{kruger2014benchmarking} and general
%                \item Requires continuous power
%            \end{itemize}
%        \end{block}
%
%        \begin{block}{RFID\footcite{want2006introduction}}
%            \begin{itemize}
%                \item Requires middleware for internet communication
%                \item Specialized on one task
%                \item No batteries or continuous power required
%            \end{itemize}
%        \end{block}
%
%    \end{frame}

    \begin{frame}{The IoT gateway}

        \begin{figure}[h!]
            \centering
            \begin{tikzpicture}[auto]
                \node [block] (device_1) {Device 1};
                \node [block, right of=device_1, node distance=3cm] (device_2) {Device 2};
                \node [node distance = 0.5cm, right = of device_2] (dots) {...};
                \node [block, node distance = 0.5cm, right = of dots] (device_n) {Device n};
                \node [block, above of=device_2, node distance=2.5cm] (gateway) {Gateway};
                \node [block, above = of gateway, node distance=2.5cm]
                (internet) {Internet};

                \path [->] (device_1) edge [bend left=5] (gateway);
                \path [->] (gateway) edge [bend left=5] (device_1);
                \path [->] (device_2) edge [bend left=5] (gateway);
                \path [->] (gateway) edge [bend left=5] (device_2);
                \path [->] (device_n) edge [bend left=5] (gateway);
                \path [->] (gateway) edge [bend left=5] (device_n);
                \path [->] (gateway) edge [bend left=5] (internet);
                \path [->] (internet) edge [bend left=5] (gateway);
            \end{tikzpicture}
        \end{figure}

    \end{frame}

    \subsection{Asynchronous I/O and libuv}

    \begin{frame}{Original idea}

        \begin{itemize}
            \item Experience from embedded gateway-like systems
            \item Is there a better approach than a synchronous loop?
            \item Can web technologies have an answer?
        \end{itemize}

    \end{frame}

    \begin{frame}{NodeJS}

        \lstinputlisting[
          language=javascript,
          caption={Synchronous I/O example in NodeJS.},
          label={listing:read_file_sync}
        ]{../thesis/figures/read_file_sync.js}

        \lstinputlisting[
          language=javascript,
          caption={Asynchronous I/O example in NodeJS.},
          label={listing:read_file_async}
        ]{../thesis/figures/read_file_async.js}

    \end{frame}

    \begin{frame}{The event loop}

        \begin{figure}[h!]
        \centering
            \begin{tikzpicture}[node distance=1.5cm, auto]
                \node [decision] (alive) {Loop alive?};
                \node [block, left of=alive, node distance=3cm] (end) {End};
                \node [block, below of=alive, node distance=2.5cm] (poll) {Poll for I/O};
                \node [block, below of=poll]
                (run_callbacks) {Run callbacks};

                \path [line] (alive) -- node {No} (end);
                \path [line] (alive) -- node {Yes} (poll);
                \path [line] (run_callbacks.east) -- ++(1cm,0) |- (alive.east);
            \end{tikzpicture}
        \end{figure}

    \end{frame}

    \begin{frame}{libuv}

        \begin{figure}
            \centering
            \resizebox{\columnwidth}{!} {
            \includegraphics{libuv.png}
            }
            \caption{The libuv logo. \textcopyright The libuv team (\url{libuv.org}).}
        \end{figure}


    \end{frame}

%    \begin{frame}{Motivation}
%
%        What is the problem?
%
%        \begin{itemize}
%            \item Kruger et al. state little work has been done to explore the
%                performance of IoT devices when different parameters are
%                adjusted \footcite{kruger2014benchmarking}.
%
%            \item There is an interest from the industry to understand and
%                increase the knowledge of IoT devices and applications.
%        \end{itemize}
%
%    \end{frame}

    \begin{frame}{Research questions}

        \begin{itemize}
            \item What internal and external properties affect the
                functionality and performance of the IoT gateway?

            \item How can libuv be used in order to implement an IoT gateway?
                What are the benefits and disadvantages of doing so?
        \end{itemize}

    \end{frame}

    \section{Identifying the gateway}

    \begin{frame}{Two customer cases}

        Customer A

        \begin{figure}[h!]
            \centering
            \resizebox{0.5\columnwidth}{!} {
            \begin{tikzpicture}[auto]
                \node [block] (device_1) {Device 1};
                \node [block, right of=device_1, node distance=3cm] (device_2) {Device 2};
                \node [node distance = 0.5cm, right = of device_2] (dots) {...};
                \node [block, node distance = 0.5cm, right = of dots] (device_n) {Device n};
                \node [block, below of=device_2, node distance=2.5cm] (gateway) {Gateway};

                \path [->] (device_1) edge [bend left=5] (gateway);
                \path [->] (gateway) edge [bend left=5] (device_1);
                \path [->] (device_2) edge [bend left=5] (gateway);
                \path [->] (gateway) edge [bend left=5] (device_2);
                \path [->] (device_n) edge [bend left=5] (gateway);
                \path [->] (gateway) edge [bend left=5] (device_n);
            \end{tikzpicture}
            }
        \end{figure}

        Customer B

        \begin{figure}[h!]
            \centering
            \resizebox{0.5\columnwidth}{!} {
            \begin{tikzpicture}[auto]
                \node [block] (device_1) {Device 1};
                \node [block, right of=device_1, node distance=3cm] (device_2) {Device 2};
                \node [node distance = 0.5cm, right = of device_2] (dots) {...};
                \node [block, node distance = 0.5cm, right = of dots] (device_n) {Device n};
                \node [block, below of=device_2, node distance=2.5cm] (gateway) {Gateway};

                \path [line] (device_1) -- (gateway);
                \path [line] (device_2) -- (gateway);
                \path [line] (device_n) -- (gateway);
            \end{tikzpicture}
            }
        \end{figure}

    \end{frame}

%    \begin{frame}{Task management}
%
 %       Discussed by Adya et al.\footcite{adya2002cooperative}.
%
 %       \begin{block}{Serial task management}
 %           Tasks are run until they are finished before any other task can
 %           start its execution.
 %       \end{block}
%
 %       \begin{block}{Preemptive task management}
 %           Tasks can be scheduled to interleave and make room for other tasks
 %           to be run concurrently.
 %       \end{block}
%
 %       \begin{block}{Cooperative task management}
 %           Tasks can yield control to other tasks in defined points in its
 %           execution. Useful in I/O intensive systems.
 %       \end{block}
%
 %   \end{frame}

    \begin{frame}{Serial order handling}

        \begin{figure}[h!]
            \centering
            \resizebox{\columnwidth}{!}{
            \begin{tikzpicture}[node distance=5cm, auto, >=stealth']
                \node[] (brewer) {Brewer};
                \node[left of=brewer] (barista) {Barista};
                \node[left of=barista] (customers) {Customer line};

                \node[below of=brewer, node distance=7.5cm] (brewer_ground) {};
                \node[below of=barista, node distance=7.5cm] (barista_ground) {};
                \node[below of=customers, node distance=7.5cm] (customers_ground) {};

                \draw (brewer) -- (brewer_ground);
                \draw (barista) -- (barista_ground);
                \draw (customers) -- (customers_ground);

                \draw[<-] ($(barista)!1cm!(barista_ground)$) --
                    node[above, scale=0.9, midway]{Place $Order_0$}
                    ($(customers)!1cm!(customers_ground)$);
                \draw[<-] ($(brewer)!1.1cm!(brewer_ground)$) --
                    node[above, scale=0.9, midway]{Start brewer}
                    ($(barista)!1.1cm!(barista_ground)$);
                \node[draw=none, fill=white, scale=0.9] at
                    ($(brewer)!1.5cm!(brewer_ground)$) {Brewing...};
                \draw[->] ($(brewer)!1.9cm!(brewer_ground)$) --
                    node[above, scale=0.9, midway]{Coffee ready}
                    ($(barista)!1.9cm!(barista_ground)$);
                \draw[->] ($(barista)!2cm!(barista_ground)$) --
                    node[above, scale=0.9, midway]{Serve $Order_0$}
                    ($(customers)!2cm!(customers_ground)$);

                \node[draw=none, fill=white, scale=0.9] at
                    ($(customers)!2.5cm!(customers_ground)$)
                    {Next customer};

                \draw[<-] ($(barista)!3cm!(barista_ground)$) --
                    node[above, scale=0.9, midway]{Place $Order_1$}
                    ($(customers)!3cm!(customers_ground)$);
                \draw[<-] ($(brewer)!3.1cm!(brewer_ground)$) --
                    node[above, scale=0.9, midway]{Start brewer}
                    ($(barista)!3.1cm!(barista_ground)$);
                \node[draw=none, fill=white, scale=0.9] at
                    ($(brewer)!3.5cm!(brewer_ground)$) {Brewing...};
                \draw[->] ($(brewer)!3.9cm!(brewer_ground)$) --
                    node[above, scale=0.9, midway]{Coffee ready}
                    ($(barista)!3.9cm!(barista_ground)$);
                \draw[->] ($(barista)!4cm!(barista_ground)$) --
                    node[above, scale=0.9, midway]{Serve $Order_1$}
                    ($(customers)!4cm!(customers_ground)$);

                \node[draw=none, fill=white, scale=0.9] at
                    ($(customers)!4.5cm!(customers_ground)$)
                    {Next customer};

                \node[draw=none, fill=white, scale=0.9] at ($(brewer)!5cm!(brewer_ground)$)
                    {...};
                \node[draw=none, fill=white, scale=0.9] at ($(barista)!5cm!(barista_ground)$)
                    {...};
                \node[draw=none, fill=white, scale=0.9] at ($(customers)!5cm!(customers_ground)$)
                    {...};

                \draw[<-] ($(barista)!6cm!(barista_ground)$) --
                    node[above, scale=0.9, midway]{Place $Order_j$}
                    ($(customers)!6cm!(customers_ground)$);
                \draw[<-] ($(brewer)!6.1cm!(brewer_ground)$) --
                    node[above, scale=0.9, midway]{Start brewer}
                    ($(barista)!6.1cm!(barista_ground)$);
                \node[draw=none, fill=white, scale=0.9] at
                    ($(brewer)!6.5cm!(brewer_ground)$) {Brewing...};
                \draw[->] ($(brewer)!6.9cm!(brewer_ground)$) --
                    node[above, scale=0.9, midway]{Coffee ready}
                    ($(barista)!6.9cm!(barista_ground)$);
                \draw[->] ($(barista)!7cm!(barista_ground)$) --
                    node[above, scale=0.9, midway]{Serve $Order_j$}
                    ($(customers)!7cm!(customers_ground)$);

                \node[draw=none, fill=white, scale=0.9] at ($(brewer)!7.5cm!(brewer_ground)$)
                    {...};
                \node[draw=none, fill=white, scale=0.9] at ($(barista)!7.5cm!(barista_ground)$)
                    {...};
                \node[draw=none, fill=white, scale=0.9] at ($(customers)!7.5cm!(customers_ground)$)
                    {...};
            \end{tikzpicture}
            }
        \end{figure}

    \end{frame}

    \begin{frame}{Preemptive order handling}

        \begin{figure}[h!]
            \centering
            \resizebox{\columnwidth}{!}{
            \begin{tikzpicture}[node distance=3cm, auto, >=stealth']
                \node[] (brewern) {$Brewer_{n-1}$};
                \node[left of=brewern] (baristan) {$Barista_{n-1}$};
                \node[left of=baristan, node distance=1.5cm] (dots) {...};
                \node[left of=dots, node distance=1.5cm] (brewer0) {$Brewer_0$};
                \node[left of=brewer0] (barista0) {$Barista_0$};
                \node[left of=barista0] (customers) {Customer line};

                \node[below of=brewern, node distance=7cm] (brewern_ground) {};
                \node[below of=baristan, node distance=7cm] (baristan_ground) {};
                \node[below of=dots, node distance=7cm] (dots_ground) {};
                \node[below of=brewer0, node distance=7cm] (brewer0_ground) {};
                \node[below of=barista0, node distance=7cm] (barista0_ground) {};
                \node[below of=customers, node distance=7cm] (customers_ground) {};

                \draw (brewern) -- (brewern_ground);
                \draw (baristan) -- (baristan_ground);
                \draw (brewer0) -- (brewer0_ground);
                \draw (barista0) -- (barista0_ground);
                \draw (customers) -- (customers_ground);

                \draw[<-] ($(barista0)!1cm!(barista0_ground)$) --
                    node[above, scale=0.9, midway]{Place $Order_0$}
                    ($(customers)!1cm!(customers_ground)$);
                \draw[<-] ($(brewer0)!1.1cm!(brewer0_ground)$) --
                    node[above, scale=0.9, midway]{Start brewer}
                    ($(barista0)!1.1cm!(barista0_ground)$);
                \node[draw=none, fill=white, scale=0.9] at
                    ($(brewer0)!1.5cm!(brewer0_ground)$) {Brewing...};
                \draw[->] ($(brewer0)!2.9cm!(brewer0_ground)$) --
                    node[above, scale=0.9, midway]{Coffee ready}
                    ($(barista0)!2.9cm!(barista0_ground)$);
                \draw[->] ($(barista0)!3cm!(barista0_ground)$) --
                    node[above, scale=0.9, midway]{Serve $Order_0$}
                    ($(customers)!3cm!(customers_ground)$);

                \node[draw=none, fill=white, scale=0.9] at
                    ($(customers)!1.5cm!(customers_ground)$)
                    {Next customer};

                \draw[<-, dashed] ($(dots)!2cm!(dots_ground)$) --
                    node[above, scale=0.9, pos=0.4]{Place $Order_1$}
                    ($(customers)!2cm!(customers_ground)$);

                \node[draw=none, fill=white, scale=0.9] at
                    ($(customers)!3.5cm!(customers_ground)$)
                    {Next customer};

                \node[draw=none, fill=white, scale=0.9] at ($(brewern)!4cm!(brewern_ground)$)
                    {...};
                \node[draw=none, fill=white, scale=0.9] at ($(baristan)!4cm!(baristan_ground)$)
                    {...};
                \node[draw=none, fill=white, scale=0.9] at ($(brewer0)!4cm!(brewer0_ground)$)
                    {...};
                \node[draw=none, fill=white, scale=0.9] at ($(barista0)!4cm!(barista0_ground)$)
                    {...};
                \node[draw=none, fill=white, scale=0.9] at ($(customers)!4cm!(customers_ground)$)
                    {...};

                \draw[<-] ($(baristan)!4.5cm!(baristan_ground)$) --
                    node[above, scale=0.9, midway]{Place $Order_j$}
                    ($(customers)!4.5cm!(customers_ground)$);
                \draw[<-] ($(brewern)!4.6cm!(brewern_ground)$) --
                    node[above, scale=0.9, midway]{Start brewer}
                    ($(baristan)!4.6cm!(baristan_ground)$);
                \node[draw=none, fill=white, scale=0.9] at
                    ($(brewern)!5.5cm!(brewern_ground)$) {Brewing...};
                \draw[->] ($(brewern)!6.4cm!(brewern_ground)$) --
                    node[above, scale=0.9, midway]{Coffee ready}
                    ($(baristan)!6.4cm!(baristan_ground)$);
                \draw[->] ($(baristan)!6.5cm!(baristan_ground)$) --
                    node[above, scale=0.9, midway]{Serve $Order_j$}
                    ($(customers)!6.5cm!(customers_ground)$);

                \draw[->, dashed] ($(dots)!5.5cm!(dots_ground)$) --
                    node[above, scale=0.9, pos=0.41]{Serve $Order_{j-1}$}
                    ($(customers)!5.5cm!(customers_ground)$);

                \node[draw=none, fill=white, scale=0.9] at ($(brewern)!7cm!(brewern_ground)$)
                    {...};
                \node[draw=none, fill=white, scale=0.9] at ($(baristan)!7cm!(baristan_ground)$)
                    {...};
                \node[draw=none, fill=white, scale=0.9] at ($(brewer0)!7cm!(brewer0_ground)$)
                    {...};
                \node[draw=none, fill=white, scale=0.9] at ($(barista0)!7cm!(barista0_ground)$)
                    {...};
                \node[draw=none, fill=white, scale=0.9] at ($(customers)!7cm!(customers_ground)$)
                    {...};
            \end{tikzpicture}
            }
        \end{figure}

    \end{frame}

    \begin{frame}{Cooperative order handling}

        \begin{figure}[h!]
            \centering
            \resizebox{\columnwidth}{!}{
            \begin{tikzpicture}[node distance=3cm, auto, >=stealth']
                \node[] (brewern) {$Brewer_{n-1}$};
                \node[left of=brewern, node distance=1.5cm] (dots) {...};
                \node[left of=dots, node distance=1.5cm] (brewer0) {$Brewer_0$};
                \node[left of=brewer0] (barista) {Barista};
                \node[left of=barista] (customers) {Customer line};

                \node[below of=brewern, node distance=7cm] (brewern_ground) {};
                \node[below of=dots, node distance=7cm] (dots_ground) {};
                \node[below of=brewer0, node distance=7cm] (brewer0_ground) {};
                \node[below of=barista, node distance=7cm] (barista_ground) {};
                \node[below of=customers, node distance=7cm] (customers_ground) {};

                \draw (brewern) -- (brewern_ground);
                \draw (brewer0) -- (brewer0_ground);
                \draw (barista) -- (barista_ground);
                \draw (customers) -- (customers_ground);

                \draw[<-] ($(barista)!1cm!(barista_ground)$) --
                    node[above, scale=0.9, midway]{Place $Order_0$}
                    ($(customers)!1cm!(customers_ground)$);
                \draw[<-] ($(brewer0)!1.1cm!(brewer0_ground)$) --
                    node[above, scale=0.9, midway]{Start brewer}
                    ($(barista)!1.1cm!(barista_ground)$);
                \node[draw=none, fill=white, scale=0.9] at
                    ($(brewer0)!1.5cm!(brewer0_ground)$) {Brewing...};
                \draw[->] ($(brewer0)!2.9cm!(brewer0_ground)$) --
                    node[above, scale=0.9, midway]{Coffee ready}
                    ($(barista)!2.9cm!(barista_ground)$);
                \draw[->] ($(barista)!3cm!(barista_ground)$) --
                    node[above, scale=0.9, midway]{Serve $Order_0$}
                    ($(customers)!3cm!(customers_ground)$);

                \node[draw=none, fill=white, scale=0.9] at
                    ($(customers)!1.5cm!(customers_ground)$)
                    {Next customer};

                \draw[<-] ($(barista)!2cm!(barista_ground)$) --
                    node[above, scale=0.9, pos=0.4]{Place $Order_1$}
                    ($(customers)!2cm!(customers_ground)$);
                \draw[<-, dashed] ($(dots)!2.1cm!(dots_ground)$) --
                    node[above, scale=0.9, pos=0.65]{Start brewer}
                    ($(barista)!2.1cm!(barista_ground)$);

                \node[draw=none, fill=white, scale=0.9] at
                    ($(customers)!3.5cm!(customers_ground)$)
                    {Next customer};

                \node[draw=none, fill=white, scale=0.9] at ($(brewern)!4cm!(brewern_ground)$)
                    {...};
                \node[draw=none, fill=white, scale=0.9] at ($(brewer0)!4cm!(brewer0_ground)$)
                    {...};
                \node[draw=none, fill=white, scale=0.9] at ($(barista)!4cm!(barista_ground)$)
                    {...};
                \node[draw=none, fill=white, scale=0.9] at ($(customers)!4cm!(customers_ground)$)
                    {...};

                \draw[<-] ($(barista)!4.5cm!(barista_ground)$) --
                    node[above, scale=0.9, midway]{Place $Order_j$}
                    ($(customers)!4.5cm!(customers_ground)$);
                \draw[<-] ($(brewern)!4.6cm!(brewern_ground)$) --
                    node[above, scale=0.9, pos=0.75]{Start brewer}
                    ($(barista)!4.6cm!(barista_ground)$);
                \node[draw=none, fill=white, scale=0.9] at
                    ($(brewern)!5.5cm!(brewern_ground)$) {Brewing...};
                \draw[->] ($(brewern)!6.4cm!(brewern_ground)$) --
                    node[above, scale=0.9, pos=0.75]{Coffee ready}
                    ($(barista)!6.4cm!(barista_ground)$);
                \draw[->] ($(barista)!6.5cm!(barista_ground)$) --
                    node[above, scale=0.9, midway]{Serve $Order_j$}
                    ($(customers)!6.5cm!(customers_ground)$);

                \draw[->, dashed] ($(dots)!5.5cm!(dots_ground)$) --
                    node[above, scale=0.9, pos=0.41]{Serve $Order_{j-1}$}
                    ($(customers)!5.5cm!(customers_ground)$);

                \node[draw=none, fill=white, scale=0.9] at ($(brewern)!7cm!(brewern_ground)$)
                    {...};
                \node[draw=none, fill=white, scale=0.9] at ($(brewer0)!7cm!(brewer0_ground)$)
                    {...};
                \node[draw=none, fill=white, scale=0.9] at ($(barista)!7cm!(barista_ground)$)
                    {...};
                \node[draw=none, fill=white, scale=0.9] at ($(customers)!7cm!(customers_ground)$)
                    {...};
            \end{tikzpicture}
            }
        \end{figure}

    \end{frame}

    \begin{frame}{A general approach}

        \begin{figure}[h!]
            \centering
            \resizebox{!}{18em}{
            \begin{tikzpicture}[node distance=5cm, auto, >=stealth']
                \node[] (event_handler) {Event handler};
                \node[left of=event_handler] (dispatcher) {Dispatcher};

                \node[below of=event_handler, node distance=8cm] (event_handler_ground) {};
                \node[below of=dispatcher, node distance=8cm] (dispatcher_ground) {};

                \draw (event_handler) -- (event_handler_ground);
                \draw (dispatcher) -- (dispatcher_ground);

                \draw[<-] ($(event_handler)!1cm!(event_handler_ground)$) --
                    node[above, scale=0.9, midway]{Dispatch $\epsilon_0$}
                    ($(dispatcher)!1cm!(dispatcher_ground)$);
                \node[draw=none, fill=white, scale=0.9] at
                    ($(event_handler)!1.5cm!(event_handler_ground)$)
                    {$\Lambda_0$};
                \draw[->] ($(event_handler)!2cm!(event_handler_ground)$) --
                    node[above, scale=0.9, midway]{Finish $\epsilon_0$}
                    ($(dispatcher)!2cm!(dispatcher_ground)$);

                \draw[<-] ($(event_handler)!3cm!(event_handler_ground)$) --
                    node[above, scale=0.9, midway]{Dispatch $\epsilon_1$}
                    ($(dispatcher)!3cm!(dispatcher_ground)$);
                \node[draw=none, fill=white, scale=0.9] at
                    ($(event_handler)!3.5cm!(event_handler_ground)$)
                    {$\Lambda_1$};
                \draw[->] ($(event_handler)!4cm!(event_handler_ground)$) --
                    node[above, scale=0.9, midway]{Finish $\epsilon_1$}
                    ($(dispatcher)!4cm!(dispatcher_ground)$);

                \node[draw=none, fill=white, scale=0.9] at
                ($(event_handler)!5cm!(event_handler_ground)$)
                    {...};
                \node[draw=none, fill=white, scale=0.9] at
                ($(dispatcher)!5cm!(dispatcher_ground)$)
                    {...};

                \draw[->] ($(event_handler)!5.5cm!(event_handler_ground)$) --
                    node[above, scale=0.9, midway]{Finish $\epsilon_{j-1}$}
                    ($(dispatcher)!5.5cm!(dispatcher_ground)$);

                \draw[<-] ($(event_handler)!6.5cm!(event_handler_ground)$) --
                    node[above, scale=0.9, midway]{Dispatch $\epsilon_j$}
                    ($(dispatcher)!6.5cm!(dispatcher_ground)$);
                \node[draw=none, fill=white, scale=0.9] at
                    ($(event_handler)!7cm!(event_handler_ground)$)
                    {$\Lambda_j$};
                \draw[->] ($(event_handler)!7.5cm!(event_handler_ground)$) --
                    node[above, scale=0.9, midway]{Finish $\epsilon_j$}
                    ($(dispatcher)!7.5cm!(dispatcher_ground)$);

                \node[draw=none, fill=white, scale=0.9] at
                ($(event_handler)!8cm!(event_handler_ground)$)
                    {...};
                \node[draw=none, fill=white, scale=0.9] at
                ($(dispatcher)!8cm!(dispatcher_ground)$)
                    {...};
            \end{tikzpicture}
            }
        \end{figure}

    \end{frame}

    \begin{frame}{Three event propagation models}

        \begin{block}{Serial event propagation}
%             Given a set of $m$ events $\{ \epsilon_0, \epsilon_1, ...,
%             \epsilon_j, ..., \epsilon_{m-1} \}$ no event $\epsilon_j$ is
%             dispatched before the work $\Lambda_{j-1}$ is finished.

            \begin{itemize}
                \item Dispatch and handle one event at a time
                \item Typically single-threaded
            \end{itemize}
        \end{block}

        \begin{block}{Preemptive event propagation}
%            Given a set of $m$ events $\{ \epsilon_0, \epsilon_1, ...,
%            \epsilon_j, ..., \epsilon_{m-1} \}$ each event $\epsilon_j$ can be
%            dispatched independent on $\Lambda_{j-1}$ and the work $\Lambda_j$
%            can \textbf{implicitly} interleave and make room for work $\Lambda_k$ where
%            $k \ne j$.

            \begin{itemize}
                \item Dispatch and process many events concurrently
                \item Typically multi-threaded
            \end{itemize}
        \end{block}

        \begin{block}{Cooperative event propagation}
%            Given a set of $m$ events $\{ \epsilon_0, \epsilon_1, ...,
%            \epsilon_j, ..., \epsilon_{m-1} \}$ each event $\epsilon_j$ can be
%            dispatched independent on $\Lambda_{j-1}$ and the work $\Lambda_j$
%            can \textbf{explicitly} make room for work $\Lambda_k$ where $k \ne j$.

            \begin{itemize}
                \item Dispatch and process many events concurrently
                \item Single-threaded, but asynchronous
            \end{itemize}
        \end{block}

    \end{frame}

    \begin{frame}{The gateway components}

        The pull-based dispatcher was chosen for implementation.

        \begin{description}
            \item[Serial dispatcher:] Fetch and dispatch one event at a time.
            \item[Preemptive dispatcher:] Use one thread per device.
            \item[Cooperative dispatcher:] Fetch and dispatch events asynchronously.
            \item[Serial event handler:] Process work $\Lambda$ serially.
            \item[Preemptive event handler:] Use one thread per work $\Lambda$.
            \item[Cooperative event handler:] Do CPU related work on one thread and I/O
                related work asynchronously.
        \end{description}

    \end{frame}

    \section{An abstract gateway}

    \begin{frame}{An abstract gateway}

        $$
        \Gamma = \big \langle \Omega, \phi, \delta, K, \Lambda, \chi \big
        \rangle
        $$

        $$
        \Omega = \{ \omega_0, \omega_1, ..., \omega_{q-1} \} \text{, }
        K = \{ \kappa_0, \kappa_1 \} \text{, }
        \Lambda = \{ \lambda_0, \lambda_1 \}
        $$

        \begin{table}[h!]
            \begin{center}
                \begin{tabular}{|l|l|}
                    \hline
                    Parameter   & Description \\
                    \hline
                    $q = |\Omega|$         & Number of devices. \\
                    $\phi$      & Frequency of generated events. \\
                    $\delta$    & The delay, or latency, added to the network. \\
                    $\kappa_0$  & Event propagation model of the dispatcher. \\
                    $\kappa_1$  & Event propagation model of the event handler. \\
                    $\lambda_0$ & CPU intensity induced by each event. \\
                    $\lambda_1$ & I/O intensity induced by each event. \\
                    $\chi$      & Number of CPU cores. \\
                    \hline
                \end{tabular}
            \end{center}
        \end{table}

    \end{frame}

    \begin{frame}{The $E$ function}

        $$
        E(\Gamma, t_{start}, t_{end}) \to M = \big[ \epsilon_0, \epsilon_1,
        ..., \epsilon_{m-1} \big]^T
        $$

        $$
        \epsilon_j = [ t_0^j, t_1^j, t_2^j, t_3^j, t_4^j ]
        $$

        \begin{itemize}

            \item $t_0^j$ represents the point in time when $\epsilon_j$ was
                created in $\omega_i$.

            \item $t_1^j$ represents the point in time when $\epsilon_j$ left
                $\omega_i$.

            \item $t_2^j$ represents the point in time when $\epsilon_j$
                arrived at $\Gamma$.

            \item $t_3^j$ represents the point in time when $\epsilon_j$ was
                dispatched to the event handler in $\Gamma$.

            \item $t_4^j$ represents the point in time when $\Gamma$ finished
                processing $\epsilon_j$.

        \end{itemize}

    \end{frame}

%    \begin{frame}{Constraints on time values}
%
%        $$
%        t_{start} \leq t_{end}
%        $$
%
%        $$
%        t_{start} \leq t_k^j
%        $$
%
%        $$
%        t_0^j \leq t_{end}
%        $$
%
%        $$
%        t_0^j \leq t_1^j \leq t_2^j \leq t_3^j \leq t_4^j
%        $$
%
%    \end{frame}

    \begin{frame}{Performance metrics}

        \[
            \pi(\epsilon_j) =
        \begin{cases}
            1, & \text{if } t_4^j \leq t_{end} \\
            0, & \text{otherwise}
        \end{cases}
        \]

        $$
        T = \frac{\sum_{\epsilon_j \in M}{\pi(\epsilon_j)}}{t_{end} -
        t_{start}} \text{ [events/tu]}
        $$

        \begin{itemize}

            \item $d_0^j = t_4^j - t_0^j$ is the response time of $\epsilon_j$,
                i.e.  the total time between that the event was created until
                it was processed.

            \item $d_1^j = t_1^j - t_0^j$ is the wait time, i.e. the time
                $\epsilon_j$ spent in $\omega_i$.

            \item $d_2^j = t_4^j - t_2^j$ is the process time, i.e. the time
                $\epsilon_j$ spent in $\Gamma$.

        \end{itemize}

    \end{frame}

%    \begin{frame}{Analyzing the load}
%
 %       A system is \textit{at load} when it is working at full capacity and is
 %       not able to produce higher throughput.
%
 %       $$
 %       P = \frac{\sum_{\epsilon_j \in M}{\pi(\epsilon_j)}}{m}
 %       $$
%
 %       $$
 %       P \simeq 1 \rightarrow \text{Maybe at load}
 %       $$
%
 %       $$
 %       P \ll 1 \rightarrow \text{Increase in response time but max throughput}
 %       $$
%
 %       A proper load is one where $P \simeq 1$, but where an increase in $m$
 %       only reduces $P$.
%
 %       $$
 %       m = q \times \phi \times (t_{end} - t_{start}) \text{, }
 %       \phi \approx \frac{T_{max}}{q}
 %       $$
%
 %   \end{frame}

    \section{Results}

    \begin{frame}{Test overview}

        \begin{itemize}
            \item[s/s:] Serial dispatcher with a serial event handler.
            \item[s/p:] Serial dispatcher with a preemptive event handler.
            \item[c/c:] Cooperative dispatcher with a cooperative event
                handler.
            \item[c/p:] Cooperative dispatcher with a preemptive event handler.
        \end{itemize}

        \bgroup
        \def\arraystretch{1.2}
        \begin{table}[h!]
        \begin{center}
        \resizebox{\columnwidth}{!} {
        \begin{tabular}{|r|r|r|r|r|r|}
            \hline
            \# & Cores $\chi$ & Quantity $q$ & Delay $\delta$ [s] & CPU int. $\lambda_0$ & I/O int. $\lambda_1$ \\
            \hline
            1 & 4 & $1 \rightarrow 50$ & 0 & 0.1 & $2 \times 10^{-4}$ \\
            \hline
            2 & 4 & 25 & $0.01 \rightarrow 0.1$ & 0.1 & $2 \times 10^{-4}$ \\
            \hline
            3 & 4 & 25 & 0 & $0.1 \rightarrow 0.5$ & $10^{-4}$ \\
            \hline
            4 & 1 & 25 & 0 & $0.1 \rightarrow 0.5$ & $10^{-4}$ \\
            \hline
            5 & 4 & 25 & 0 & 0.05 & $10^{-4} \rightarrow 10^{-3}$ \\
            \hline
            6 & 1 & 25 & 0 & 0.05 & $10^{-4} \rightarrow 10^{-3}$ \\
            \hline
        \end{tabular}
        }
        \end{center}
        \end{table}

    \end{frame}

    \subsection{Test case 1}

%    \begin{frame}{Test case 1: Increasing device quantity}
%
%        \begin{figure}[h!]
%            \centering
%            \performanceplot
%            {../thesis/data/quad_quantity_throughput.csv}
%            {
%                xlabel=Device quantity $q$,
%                ylabel={Throughput [events/s]},
%                legend pos=outer north east
%            }
%        \end{figure}
%
%    \end{frame}

    \begin{frame}{Test case 1: Increasing device quantity}

        \begin{figure}[h!]
            \centering
            \performanceplot
            {../thesis/data/quad_quantity_avg_d0.csv}
            {
                xlabel=Device quantity $q$,
                ylabel={Response time [ms]},
                legend pos=outer north east
            }
        \end{figure}

    \end{frame}

    \subsection{Test case 2}

%    \begin{frame}{Test case 2: Increasing network delay}
%
%        \begin{figure}[h!]
%            \centering
%            \performanceplot
%            {../thesis/data/quad_delay_throughput.csv}
%            {
%                xlabel={Network delay $\delta$ [s]},
%                ylabel={Throughput [events/s]},
%                legend pos=outer north east
%            }
%        \end{figure}
%
%    \end{frame}

    \begin{frame}{Test case 2: Increasing network delay}

        \begin{figure}[h!]
            \centering
            \performanceplot
            {../thesis/data/quad_delay_avg_d0.csv}
            {
                xlabel={Network delay $\delta$ [s]},
                ylabel={Response time [ms]},
                legend pos=outer north east
            }
        \end{figure}

    \end{frame}

    \subsection{Test case 3 and 4}

    \begin{frame}{Test case 3 and 4: Increasing CPU intensity}

        \begin{figure}[h!]
            \centering
            \begin{tikzpicture}
                \begin{groupplot}[
                        group style = {group size = 2 by 1},
                        width = 0.5*\textwidth,
                        xlabel = CPU intensity $\lambda_0$,
                        ymax = 125,
                        legend columns = -1,
                        legend style = { at = {(-0.3, -0.6)}, anchor = north }
                    ]
                    \nextperformanceplot{../thesis/data/quad_cpu_throughput.csv}{
                        title = Quad core CPU,
                        ylabel = {Throughput [events/s]}
                    }
                    \nextperformanceplotwithlegend{../thesis/data/single_cpu_throughput.csv}{
                        title = Single core CPU
                    }
                \end{groupplot}
            \end{tikzpicture}
        \end{figure}

    \end{frame}

%    \begin{frame}{Test case 3 and 4: Increasing CPU intensity}
%
%        \begin{figure}[h!]
%            \centering
%            \begin{tikzpicture}
%                \begin{groupplot}[
%                        group style = {group size = 2 by 1},
%                        width = 0.5*\textwidth,
%                        xlabel = CPU intensity $\lambda_0$,
%                        legend columns = -1,
%                        legend style = { at = {(-0.3, -0.6)}, anchor = north }
%                    ]
%                    \nextperformanceplot{../thesis/data/quad_cpu_avg_d0.csv}{
%                        title = Quad core CPU,
%                        ylabel = {Response time [ms]}
%                    }
%                    \nextperformanceplotwithlegend{../thesis/data/single_cpu_avg_d0.csv}{
%                        title = Single core CPU
%                    }
%                \end{groupplot}
%            \end{tikzpicture}
%        \end{figure}
%
%    \end{frame}

    \subsection{Test case 5 and 6}

    \begin{frame}{Test case 5 and 6: Increasing I/O intensity}

        \begin{figure}[h!]
            \centering
            \begin{tikzpicture}
                \begin{groupplot}[
                        group style = {group size = 2 by 1},
                        width = 0.45*\textwidth,
                        xlabel = I/O intensity $\lambda_1$,
                        ymax = 125,
                        legend columns = -1,
                        legend style = { at = {(-0.3, -0.6)}, anchor = north }
                    ]
                    \nextperformanceplot{../thesis/data/quad_io_throughput.csv}{
                        ylabel = {Throughput [events/s]},
                        title = Quad core CPU
                    }
                    \nextperformanceplotwithlegend{../thesis/data/single_io_throughput.csv}{
                        title = Single core CPU
                    }
                \end{groupplot}
            \end{tikzpicture}
        \end{figure}

    \end{frame}

    \section{Conclusion}

    \begin{frame}{Conclusion}

        \begin{itemize}
            \item The abstract gateway is proposed to be able to model \textit{any}
                gateway implementation and performance tests.
            \item A number of internal and external properties of the gateway
                has been identified.
        \end{itemize}

    \end{frame}

%    \begin{frame}{Conclusion}
%
%        \begin{itemize}
%            \item Increasing the number of devices will increase response time
%                for serial and cooperative event handlers.
%            \item The cooperative dispatcher results in best response time when
%                network delay is increased.
%            \item All event propagation models perform about equally on a
%                single core system when CPU intensity is increased.
%            \item For I/O intensive work, the cooperative approaches perform
%                best in terms of throughput.
%        \end{itemize}
%
%    \end{frame}

    \begin{frame}{Conclusion}

        \begin{itemize}
            \item libuv can be used to implement a cooperative dispatcher with
                a preemptive event handler.
            \item It performs $\approx$ better or equal to the next best event
                propagation model in all test cases.
        \end{itemize}

    \end{frame}

    \begin{frame}{If I had more time}

        \begin{itemize}
            \item Implement a preemptive dispatcher.
            \item Use active devices, i.e. a push-based dispatcher design.
        \end{itemize}

    \end{frame}

    \begin{frame}
        Thank you.
    \end{frame}

    \begin{frame}
        \centering
        \resizebox{18em}{!}{
            \includegraphics{thesis_defense.png}
        }
        \footnote{\url{https://imgs.xkcd.com/comics/thesis_defense.png}}
    \end{frame}

    \begin{frame}{References}
        \printbibliography
    \end{frame}

\end{document}
