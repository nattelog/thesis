\documentclass{memoir}

\begin{document}

\paragraph{Welcome}

Welcome to the presentation of my master's thesis. My name is Natanael Log and
I am an offspring of the D-programme here at the Linköping University and I
have been conducting my master's thesis this spring at a company called
Attentec, here in Mjärdevi.

The title of my work is "A study on the performance and architectural
characteristics of the Internet of Things gateway". I have studied what is
called an Internet of Things (or IoT) gateway; its internal and external
properties and functionalities and how they affect its performance given a
certain environment it resides in.

\paragraph{Presentation outline}

\begin{description}
    \item[Background:] Sense of context
    \item[Identifying the gateway:] Method of identifying internal and external
        properties, common features
    \item[An abstract gateway:] A theoretical model that is proposed to mimic
        \textit{any} gateway implementation
    \item[Results:] A total of 24 tests
    \item[Conclusion:]
\end{description}

\paragraph{Internet of things}

Basically things connected to the internet. Basic examples. Check on your phone
if lights are off, or oven is off. Factory machines talks to each other.
Location independence. Using the internet as infrastructure.

% Quote from Uckelmann et al. Emphasize "right quantity, right time and place".
% Many definitions.

\paragraph{IoT architecture}

Three domains.

\begin{description}
    \item[Application domain:] Highest. Attentec's customer. Monitoring of industrial
        batteries.
    \item[Sensing domain:] Lowest. Devices and "things". More on the next
        slide. Converts physical attributes to digital values, e.g. a AD
        converter.
    \item[Network domain:] Middleware. The internet infrastructure. Gateways.
        Why can't devices talk directly to the internet? Some devices are not
        designed to communicate with the IP technology. But they are designed
        to communicate in other mediums. The middleware is needed to convert or
        wrap the data from the devices to the internet.
\end{description}

%\paragraph{An IoT device example}
%
%\begin{description}
%
%    \item[Raspberry Pi:] A small computer. Can communicate. Relatively powerful
%        and general.  Requires continuous power, must be turned on.
%
%    \item[RFID:] Basically an antenna with a chip. Very small. Magnetic field
%        induces current that creates a radio signal with an ID code in it. Uses
%        some sort of middleware to transfer that signal to an internet service.
%        Passive device. Key-tag example to unlock door.
%
%\end{description}

\paragraph{The IoT gateway}

The bridge between the devices and the internet.

\paragraph{Original idea}

\begin{itemize}
    \item Experience from embedded gateway-like systems. Worked with a embedded
        system that listened to activity on its peripheral serial ports, e.g.
        the CAN-bus and the UART. It piped the data to an internet service via
        Bluetooth. Explain the loop-approach. Did not scale well when the
        traffic and number of peripheral units was increased.
    \item Is there a better approach than a synchronous loop? Especially on
        small, resource-constrained systems.
    \item Can web technologies have an answer? NodeJS is a Javascript runtime
        platform that allows asynchronous operations to be done on the
        operating system API. For instance file system and network. It has
        great performance on I/O intensive systems.
\end{itemize}

\paragraph{NodeJS}

\begin{description}
    \item[Sync example:] Blocking code. Single-threaded.
    \item[Async example:] Nonblocking code. Returns instantly. Requires a
        callback. End of code will be printed before the file contents.
\end{description}

\paragraph{The event loop}

Simplified version of the event loop. An asynchronous loop. The operating
system supports functionalities to listen for activiy on each port, or
\textit{file descriptor}, and notifies the program when something has happened.

This could be a good approach to implement an IoT gateway. But is Javascript
suitable for embedded devices? The most common language is C, as it allows the
programmer to access low level components of the hardware while maintaining a
high level semantics.

\paragraph{libuv}

Fortunately, the library in NodeJS that implements the event loop is written in
C: libuv. libuv simplifies asynchronous I/O. Can libuv be used to implement a
well performant IoT gateway?

\paragraph{Motivation}

% What is the problem? Is there a problem with the gateways the way they are now?
% No, gateways exists everywhere and they seem to work fine.

% \begin{itemize}
%    \item Kruger et al. state little work has been done to explore the
%        performance of IoT devices (including gateways) when different
%        parameters are adjusted \footcite{kruger2014benchmarking}.
%
%    \item There is an interest from the industry to understand and
%        increase the knowledge of IoT devices and applications.
%\end{itemize}

What are the questions?

\begin{itemize}
    \item What internal and external properties affect the
        functionality and performance of the IoT gateway?

    \item How can libuv be used in order to implement an IoT gateway?
        What are the benefits and disadvantages of doing so?
\end{itemize}

\paragraph{Two customer cases}

Attentec, the company the thesis was conducted, has two customers that provide
IoT solutions and gateway implementations. One of them, Customer A, uses their
IoT gateway to monitor industrial batteries used in forklifts.

\begin{itemize}
    \item Pull-based
    \item Passive devices
    \item Devices do not know of the gateway, basically servers listening for
        requests.
    \item Gateway must know of the devices
\end{itemize}

Customer B uses a \textit{push-based} approach to handle event propagation. The
gateway hosts a REST API to serve IoT devices.

\begin{itemize}
    \item Push-based
    \item Active devices
    \item Devices must know the address of the gateway
    \item Gateway does not need to know about the devices
\end{itemize}

Important findings:

\begin{itemize}
    \item Gateway have either a pull- or push-based approach to handle devices.
        Maybe a combination can exist.
    \item The main architecture is a typical server-client one. Many devices,
        one gateway.
    \item The devices talk to the gateway via some medium. Can be radio, can be
        TCP based.
\end{itemize}

%\paragraph{Task management}
%
%Three ways of handling tasks. A task can be a function or a piece of code.
%
%\begin{description}
%    \item[Serial:] typically single threaded.
%    \item[Preemptive:] typically multi-threaded.
%    \item[Cooperative:] typically asynchronous.
%\end{description}
%
%These models have been transformed to better model event propagation in the
%gateway.

\paragraph{Coffe shop analogies}

Discuss each one of them.

\paragraph{A general approach}

Common entities:

\begin{description}
    \item[Orders:] events $\epsilon$
    \item[Customer line:] dispatcher, handles incoming events
    \item[Barista and brewer:] event handler, put work on each event
    \item[Work:] $\Lambda$
\end{description}

Example of a serial event propagation model.

\paragraph{Three event propagation models}

\begin{description}
    \item[Implicit:] Interleaving performed by the operating system scheduler
    \item[Explicit:] Interleaving performed by the user
\end{description}

\paragraph{The gateway components}

Pull-based dispatcher design was chosen. Explain the different combinations.

\paragraph{An abstract gateway}

A gateway $\Gamma$ is a six-tuple describing its internal properties and its
environment.

Describe each item.

CPU intensity: CPU related work. I/O intensity: e.g. reading data to a file or
sending data over the network.

These are not all possible properties. For instance push- or pull-based
dispatcher design is omitted.

\paragraph{The $E$ function}

Describe it. $\Gamma$ can be seen as static source code, while $E$ is the
runtime version of that source code.

Timestamps represents part of the event lifecycle.

% \paragraph{Constraints on time values}

\paragraph{Performance metrics}

%\paragraph{Analyzing the load}
%
%Why can't we just stick with a load $P \ll 1$?

\paragraph{Test overview}

Four of nine possible dispatcher/event handler combinations were implemented.

For each architecture, 6 tests were conducted. $\phi$ was used to get a good
load.

\paragraph{Test case 1}

\begin{description}
%    \item[Throughput:] Preemptive event handlers fastest. Why?
    \item[Response time:] Two groups. Preemptive fastest. Outliers correlated
        with load. Why?
    \item[Conclusion:] Response time scalability dependent on the event handler
        event propagation model.
\end{description}

\paragraph{Test case 2}

\begin{description}
%    \item[Throughput:] Cooperative dispatchers fastest. Why?
    \item[Response time:] Two groups. Cooperative fastest. Why?
    \item[Conclusion:] Response time scalability dependent on the dispatcher
        design.
\end{description}

\paragraph{Test case 3 and 4}

\begin{description}
    \item[Throughput:] Preemptive EH fastest for quad core. All perform about
        equally for single core. S/P has bottleneck for low CPU values.
%    \item[Response time:] Preemptive EH fastest for quad core. C/C slowest
%        response time. S/S fastest for single core.
    \item[Conclusion:] Preemptive best for multicore systems. S/S sufficient
        for single core systems.
\end{description}

\paragraph{Test case 5 and 6}

\begin{description}
    \item[Throughput:] $\approx$ same result independent on cores. Preemptive
        and cooperative fastest. Thread scheduling optimized for I/O
        operations.
    \item[Conclusion:] Cooperative approaches best. No need for multithreading.
\end{description}

\paragraph{Conclusion}

\begin{itemize}
    \item The abstract gateway is proposed to be able to model \textit{any}
        gateway implementation and performance tests.
    \item A number of internal and external properties of the gateway
        has been identified. (next slide)
    \item Increasing the number of devices will increase response time
        for serial and cooperative event handlers.
    \item The cooperative dispatcher results in best response time when
        network delay is increased.
    \item All event propagation models perform about equally on a
        single core system when CPU intensity is increased.
    \item For I/O intensive work, the cooperative approaches perform
        best in terms of throughput. (next slide)
    \item libuv can be used to implement a cooperative dispatcher with
        a preemptive event handler.
    \item It performs $\approx$ better or equal to the next best event
        propagation model in all test cases.
\end{itemize}

\paragraph{If I had more time}

\begin{itemize}
    \item Implement a preemptive dispatcher and more combinations of the event
        propagation models.
    \item Use active devices, i.e. a push-based dispatcher design.
\end{itemize}

\paragraph{Thank you}

That was the end of my presentation. There is more in the report that I haven't
brought up here due to time.

\end{document}
